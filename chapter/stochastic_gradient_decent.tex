% !TEX root = ../Masterthesis.tex

\chapter{Stochastic Gradient Decent (SGD)}

Until now we have always assumed we had access to the theoretical loss \(\Loss\) 
directly. In reality we almost never have. What we can do instead is use the
stochastic gradient
%
\begin{align*}
	\nabla_\weights\loss (\weights, X,Y) \qquad (X,Y)\sim\dist.
\end{align*}
%
Its expected value is the theoretical loss again\footnote{
	We need to be able to swap the expectation with differentiation. A sufficient 
	condition is continuity of \(\nabla\loss\) in \(\weights\), which allows us
	to use the mean value theorem
	\begin{align*}
		\frac{\partial}{\partial \weights_k}\E[\loss(\weights, X,Y)]
		&= \lim_{n\to\infty}
		\int\frac{\loss(\weights+\stdBasis_k/n, X,Y)-\loss(\weights,X,Y)}{1/n}d\Pr
		\\
		&=\lim_{n\to\infty} \int\frac{\partial}{\partial \weights_k}\loss(\xi_n, X,Y)d\Pr
		\qquad \xi_n \in [\weights, \weights + \stdBasis_k/n].
	\end{align*}
	Then we use the boundedness of a continuous function on the compact interval
	\([\weights, \weights + \stdBasis_k/n]\) to move the limit in using
	dominated convergence.
}. To shorten notation we will assume that the gradient is always meant
with respect to \(\weights\) unless otherwise stated.
Sampling an independent sequence \(((X_k,Y_k), k\ge 1)\) from the 
distribution \(\dist\) results in \emph{Stochastic Gradient Decent}
%
\begin{align*}
	\Weights_{n+1}
	&= \Weights_n - \lr_n\nabla\loss(\Weights_n, X_{n+1},Y_{n+1})\\
	&= \Weights_n - \lr_n\nabla\Loss(\Weights_n)
	+ \lr_n\underbrace{
		[\nabla\Loss(\Weights_n) - \nabla\loss(\Weights_n, X_{n+1}, Y_{n+1})]
	}_{=:\martIncr_{n+1}}
\end{align*}
Where \(\martIncr_n\) are martingale increments for the filtration
% \begin{align*}
% 	\martingale_n := \martingale_{n-1} + \martIncr_n \qquad \martingale_0:=0
% \end{align*}
\begin{align*}
	\filtration_n :=\sigma((X_k, Y_k): k\le n )
	\qquad (X_k,Y_k)\stackrel{\text{iid}}{\sim}\dist,
\end{align*}
since \(\Weights_n\) is \(\filtration_n\)-measurable which is independent of, i.e.
\((X_{n+1},Y_{n+1})\)
\begin{align}\label{eq: conditional independence of martingale increments}
	\E[\martIncr_{n+1}\mid \filtration_n]
	= \E[\nabla\Loss(\Weights_n) - \nabla\loss(\Weights_n, X_{n+1}, Y_{n+1})\mid \filtration_n]
	= 0.
\end{align}
%
\section{ODE View}

The first way to view SGD is through the lense of approximating the ODE with
integral equation
\begin{align*}
	\weights(t) = \phi^{t-t_0}\weights_0 =  \weights_0 -\int_{t_0}^{t}\nabla\Loss(\weights(s))ds.
\end{align*}
Its Euler discretization is gradient decent 
\begin{align*}
	\weights_{t_{n+1}}:=\weights_{n+1}
	= \psi^{\lr_n} \weights_n = \weights_n - \lr_n \nabla\Loss(\weights_n),
\end{align*}
where
\begin{align*}
	t_0 \le \dots \le t_N= T \quad \text{with} \quad t_{n+1}-t_n=\lr_n
	\quad \text{and}\quad \lr:=\max_n \lr_n
\end{align*}
and constant learning rate \(\lr\) represents an equidistant discretization.
Since \(\nabla\Loss\) is Lipschitz continuous, the local discretization error
\begin{align*}
	\|\phi^\lr \weights  - \psi^\lr \weights \|
	&= \|(\weights - \int_0^\lr\nabla\Loss(\phi^s \weights)ds) - (\weights-\lr\nabla\Loss(w))\|\\
	&\le  \int_0^\lr \underbrace{\|\nabla\Loss(w) - \nabla\Loss(\phi^s\weights)\|}_{
		\le \ubound \|\phi^s \weights - \weights\| \in O(s) \mathrlap{\quad\text{(Taylor approx., Lip. \(\nabla\Loss\))}}
	}ds 
\end{align*}
is of order \(O(\lr^2)\). 
Using the discrete Gr\"onwall's inequality and stability of the ODE (due to Lipschitz continuity of
\(\nabla\Loss\) and the continuous Gr\"onwall inequality) the global discretization
error is therefore of order 1, i.e.
\begin{align*}
	\max_{n} \|\weights(t_n) - \weights_n\| \in O(\lr),
\end{align*}
\fxnote{appendix?}{as is usually covered in ``Numerics of ODE's'' courses.}
Due to \((a+b)^2\le2(a^2+b^2)\), and therefore 
\begin{align*}
	\max_n\E[\|\weights(t_n)-\Weights_n\|^2]
	\le 2(\underbrace{\max_n \|\weights(t_n)-\weights_n\|^2}_{O(\lr^2)}
	+\max_{n}\E[\|\weights_n - W_n\|^2]),
\end{align*}
it is enough to bound the distance between SGD and GD.
\begin{theorem}\label{thm: distance SGD vs GD}
	In general we have
	\begin{align*}
		\E\|\weights(t_n)-\Weights_n\|^2
		\le \sum_{k=0}^{n-1}\lr_k^2\E\|\martIncr_{k+1}\|^2\exp\left(
			\sum_{j=k+1}^{n-1}\lr_j^2\ubound^2 + 2\lr_j\ubound
		\right)
	\end{align*}
	which can be simplified for constant learning rates \(\lr_n=\lr\) and bounded
	martingale increment variances \(\E\|\martIncr_n\|^2 \le K\) to
	\begin{align*}
		\max_{n}\E\|\weights(t_n)	-\Weights_n\|^2
		\le \lr TK\exp[T(\lr\ubound^2 + 2\ubound)] \in O(\lr)
	\end{align*}
\end{theorem}
Now before we get to the proof let us build some intuition why this result
is not surprising. Unrolling SGD we can rewrite it as
\begin{align*}
	\Weights_{t_n}
	= \weights_0 - \sum_{k=0}^{n-1} \lr_k \nabla\Loss(\Weights_k)
	+ \sum_{k=0}^{n-1} \lr_k\martIncr_{k+1}.
\end{align*}
Which is the same recursion as in GD except for the last term. Now for an
equidistance grid we have \(\lr=T/N\). Intuitively the last term in \(\Weights_T\)
should therefore disappear due to some law of large numbers for martingales
\begin{align}\label{eq: mean of martingale increments}
	\sum_{k=0}^{N-1}\lr\martIncr_{k+1} = \frac{T}{N}\sum_{k=1}^N\martIncr_k.
\end{align}
And since the variance of a sum of martingale increments is the sum of variances
as mixed terms disappear due to conditional independence (\ref{eq: conditional
independence of martingale increments})
\begin{align*}
	\E\left\|\sum_{k=1}^N\martIncr_k\right\|^2 = \sum_{k=1}^N \E\|\martIncr_k\|^2,
\end{align*}
the variance of a mean decreases with rate \(O(1/N)=O(\lr)\), which is the rate
we get in Theorem~\ref{thm: distance SGD vs GD}.
\begin{proof}[Proof (Theorem~\ref{thm: distance SGD vs GD})]
	Using the conditional independence (\ref{eq: conditional independence of martingale increments})
	we can get rid of a all the mixed terms with
	\(\E[\langle \cdot, \martIncr_{n+1}\rangle\mid\filtration_n]=0\) as everything
	else is \(\filtration_n\) measurable:
	\begin{align*}
		&\E\|\Weights_{n+1}-\weights_{n+1}\|^2
		= \E\|\Weights_n - \weights_n
		+ \lr_n(\nabla\Loss(\weights_n)-\nabla\Loss(\Weights_n))
		+ \lr_n\martIncr_{n+1}\|^2\\
		&\lxeq{(\ref{eq: conditional independence of martingale increments})}
		\begin{aligned}[t]
			&\E\|\Weights_n-\weights_n\|^2 + \lr_n^2\E\|\martIncr_{n+1}\|^2\\
			&+\underbrace{
				\lr_n^2\E\|\nabla\Loss(\weights_n)-\nabla\Loss(\Weights_n)\|^2
				+ 2\lr_n\E\langle\Weights_n-\weights_n,
				\nabla\Loss(\weights_n)-\nabla\Loss(\Weights_n)\rangle
			}_{
				\le 0 \qquad
				\text{for convex \(\Loss\) (Lemma~\ref{lem: bermanDiv lower bound}) and }
				\lr_n<2/\ubound
			}
		\end{aligned}
	\end{align*}
	But since we do not want to demand convexity of \(\Loss\) just yet, we will
	have to get rid of these terms in a less elegant fashion using Lipschitz
	continuity and the Cauchy-Schwarz inequality
	\begin{align*}
		\|\nabla\Loss(\weights_n)-\nabla\Loss(\Weights_n)\|^2
		&\le \ubound^2\|\weights_n-\Weights_n\|^2\\
		\langle\Weights_n-\weights_n,
		\nabla\Loss(\weights_n)-\nabla\Loss(\Weights_n)\rangle
		&\le \ubound \|\weights_n-\Weights_n\|^2
	\end{align*}
	which leads us to
	\begin{align*}
		\E\|\Weights_{n+1}-\weights_{n+1}\|^2
		\le (1+\lr_n^2\ubound^2 + 2\lr_n\ubound)\E\|\Weights_n-\weights_n\|^2
		+ \lr_n^2\E\|\martIncr_{n+1}\|^2.
	\end{align*}
	Now we apply the discrete Gr\"onwall inequality\fxnote{appendix} to get
	\begin{align*}
		\E\|\Weights_n-\weights_n\|^2
		\le \sum_{k=0}^{n-1}\lr_k^2\E\|\martIncr_k\|^2
		\underbrace{
			\prod_{j=k+1}^{n-1}(1+\lr_j^2\ubound^2 + 2\lr_j\ubound)
		}_{
			\le \exp\left(\sum_{j=k+1}^{n-1}\lr_j^2\ubound^2 + 2\lr_j\ubound\right)
		}
	\end{align*}
	using \(1+x\le\exp(x)\) for the first claim. The rest 
	follows from \(n\lr\le T\).
\end{proof}

\section{Batch Learning}

Instead of using \(\nabla\loss(\Weights_n, X_{n+1}, Y_{n+1})\) at time \(n\) as an
estimator for \(\nabla\Loss(\Weights_n)\) we could use the average of a
``batch'' of data \((X^{(i)}_{n+1}, Y^{(i)}_{n+1})_{i=1,\dots,m}\) (independently
\(\dist\) distributed) instead, i.e.
\begin{align*}
	\Weights_{n+1} = \Weights_n
	-\lr_n \underbrace{\frac1{m}\sum_{i=1}^m\nabla\loss(\Weights_n,X^{(i)}_{n+1}, Y^{(i)}_{n+1})}_{
		=:\nabla\loss_{n+1}^m(\Weights_n)
	}.
\end{align*}
leading to the modified martingale increment
\begin{align*}
	\martIncr_n^{(m)}
	= \nabla\Loss(\Weights_{n-1})
	- \nabla\loss_n^m(\Weights_{n-1})
\end{align*}
with reduced variance
\begin{align*}
	\E\|\martIncr_n^{(m)}\|^2 = \tfrac1m\E\|\martIncr_n^{(1)}\|^2 \le \tfrac1m K.
\end{align*}
Now while this variance reduction does reduce our upper bound on the distance
between SGD and GD (cf. Theorem~\ref{thm: distance SGD vs GD}), this reduction
comes at a cost: We now have to do \(m\) gradient evaluations per iteration!
We would incur an equivalent cost, if we reduced our discretization to
\begin{align*}
	\tilde{\lr}:=\frac{\lr}{m}=\frac{T}{Nm}
\end{align*}
which increases the number of iterations to reach \(T\) to \(Nm\) which implies
\(Nm\) gradient evaluations using SGD without batches. And if we have a look
at the upper bound in Theorem~\ref{thm: distance SGD vs GD} these two actions
have the same effect on our distance bound between SGD and GD (especially if
\(\Loss\) is convex as we can then get rid of the exponential term as hinted
at in the proof of the theorem).

Now of course an upper bound is no guarantee that there is no effect in reality.
But if we have a look at the term (\ref{eq: mean of martingale increments})
again which is in some sense the difference between SGD and GD, then we can
see that these actions have quite a similar effect
\begin{align*}
	\lr\sum_{k=1}^N\martIncr_k^{(m)}
	&=\frac{T}{N}\sum_{k=1}^N
	\nabla\Loss(\Weights_{k-1})- \nabla\loss_n^m(\Weights_{k-1})\\
	&=\frac{T}{Nm}\sum_{k=1}^N\sum_{i=1}^m
	[\nabla\Loss(\Weights_{k-1})-\nabla\loss(\Weights_{k-1},X^{(i)}_k, Y^{(i)}_k)]\\
	&\approx \tilde{\lr} \sum_{k=1}^{Nm}
	\underbrace{
		[\nabla\Loss(\Weights_{k-1})-\nabla\loss(\Weights_{k-1},X_k, Y_k)]
	}_{\widetilde{\martIncr}_k}.
\end{align*}
The difference is, that batch learning stays on some \(\Weights_k\) and collects
\(m\) pieces of information before it makes a big step based on this information
while SGD just makes \(m\) small steps. This also changes the true gradient
evaluations \(\nabla\Loss(\Weights_k)\) slightly. But since we are only making
small steps these \(\Weights_k\) will generally be similar. We still can not
make a definitive assertion which one is better though\footnote{
	Whether or not you think one method is superior might be correlated with
	whether or not you think that agile software development is a good idea
}.

Assuming they have the same effect on the deviation of SGD from GD we can
make a few practical considerations:
\begin{itemize}
	\item A smaller learning rate benefits GD \emph{and} reduces the distance
	between GD and SGD while batches only do the latter. If we have little data
	and want to achieve good results in as few gradient evaluations as possible
	we should in general choose smaller learning rates over larger batch sizes.
	\item SGD can not be easily parallelized since one needs the previous weights
	to calculate the next weights. Making multiple gradient evaluations at
	the same point is trivially parallelizable on the other hand and only
	requires broadcasting the result for summation. So if the bottleneck is
	computation time, it makes sense to select batch sizes equal to the number of
	threads, or if the broadcasting and summing takes significant time, batch
	sizes which are multiples of the number of threads.

	This is not without caveats though: Increasing the Batch Size to speed up
	learning requires a proportional increase in the learning rate to actually
	have an impact on the time required for optimization. And since GD stops
	converging if learing rates become too large, there is an upper bound on this.
	And while batches might be ``free'' in a computational sense due to
	parallelization, they still increase the sample consumption. And if we only
	have limited data, this often results in more passes over the same data and
	thus overfitting.
	
	Last but not least there might be other opportunities for parallelization
	like generating ensemble models by optimizing from different starting points
	or using different models altogether.
\end{itemize}
In summary: If you are concerned about the \emph{quality} of your model you should
probably choose SGD without batches, while concerns about training time justify
larger batch sizes in some cases.

\textcite{hardtTrainFasterGeneralize2016} \textcite{hofferTrainLongerGeneralize2018}

\section{Quadratic Loss Functions}

While we have proven SGD behaves roughly like GD for small learning rates, we
are not really that interested in how close SGD is to GD but rather how good it
is at optimizing \(\Loss\). So to build some intuition let us consider a
quadratic Loss function again before we get to general convex functions.
Using the same trick we used in Section~\ref{sec: visualize gd} 
\begin{align*}
	\nabla\Loss(\weights)
	= \nabla^2\Loss(\weights)(\weights-\weights_*)
	= H(\weights-\weights_*)
\end{align*}
we can rewrite SGD (by induction using the triviality of \(n=0\)) as
\begin{align}
	\nonumber
	&\Weights_{n+1}-\weights_*\\
	\nonumber
	&= \Weights_n - \weights_* - \lr_n H(\Weights_n-\weights_*) + \lr_n\martIncr_{n+1}\\
	\nonumber
	&=(1-\lr_n H)\underbrace{(\Weights_n - \weights_*)}_{
		\xeq{\text{ind.}} (\weights_n - \weights_*)
		\mathrlap{+ \sum_{k=0}^{n-1}\lr_k\left(\prod_{i=k+1}^{n-1}(1-\lr_iH)\right)\martIncr_{k+1}}
	} + \lr_n\martIncr_{n+1}\\
	\nonumber
	&= \underbrace{(1-\lr_n H)(\weights_n - \weights_*)}_{=\weights_{n+1}-\weights_* \implies (n\to n+1)}
	+ \sum_{k=0}^n\lr_k\left(\prod_{i=k+1}^n(1-\lr_iH)\right)\martIncr_{k+1}\\
	\label{eq: unrolled SGD weights (general quadratic loss case)}
	&=\left(\prod_{k=0}^n(1-\lr_kH)\right)(\weights_0-\weights_*)
	+ \sum_{k=0}^n\lr_k\left(\prod_{i=k+1}^n(1-\lr_iH)\right)\martIncr_{k+1}.
\end{align}

\subsubsection{Variance Reduction}

One particularly interesting case is \(H=\identity\). Why is that interesting?
Considering that
\begin{align*}
	\E[\tfrac12\|\weights-X\|^2]=\E[\loss(\weights,X)]=:\Loss(\weights)
\end{align*}
is minimized by \(\weights_*=\E[X]\) and we have
\begin{align*}
	\E[\nabla\loss(\weights,X)]
	=\E[\weights-X]=\weights- \weights_*=\nabla\Loss(\weights)
\end{align*}
we necessarily also have
\begin{align*}
	\Loss(\weights) = \tfrac12\|\weights-\weights_*\|^2 + \text{const},
\end{align*}
which implies \(H=\nabla^2\Loss=\identity\). So this case is essentially
trying to find the expected value of some random variables as quickly as
possible. Now by (\ref{eq: unrolled SGD weights (general
quadratic loss case)}) and
\begin{align*}
	\martIncr_n = \nabla\Loss(\Weights_{n-1}) -\nabla\loss(\weights_{n-1}, X_n)
	= X_n - \E[X]
\end{align*}
the \(\Weights_n\) are just an affine transformation of the \(X_n\). And we
happen to know the best linear unbiased estimator (``BLUE'') of \(\E[X]\)
which is the mean of the \(X_n\). This means that \(\Weights_n\) is
necessarily worse that the mean. But with \(\lr_n=\frac1{n+1}\) and by
induction
\begin{align}
	\label{eq: getting rid of w_0 first step}
	\Weights_1 &= \weights_0 - \frac{1}{0+1}(\weights_0 - X_1) = X_1\\
	\nonumber
	\Weights_{n+1}
	&= \Weights_n - \tfrac{1}{n+1}(\Weights_n - X_{n+1})
	\xeq{\text{ind.}}\smash{\underbrace{\left(1-\tfrac1{n+1}\right)}_{=\frac{n}{n+1}}}
	\frac1n\sum_{k=1}^n X_k + \tfrac{1}{n+1}X_{n+1}\\
	\nonumber
	&= \frac{1}{n+1}\sum_{k=1}^{n+1}X_k
\end{align}
we can also achieve this optimum. For constant learning rates \(\lr<1\) on the
other hand one can similarly show that we have
\begin{align*}
	\Weights_n = (1-\lr)^n\weights_0 + \sum_{k=0}^{n-1}\lr(1-\lr)^{n-1-k}X_{k+1}
\end{align*}
or to highlight the connection to (\ref{eq: unrolled SGD weights (general
quadratic loss case)})
\begin{align*}
	\Weights_n - \weights_*
	= \underbrace{(1-\lr)^n(\weights_0 -\weights_*)}_{=\weights_n-\weights_*}
	+ \sum_{k=0}^{n-1}\lr(1-\lr)^{n-1-k}
	\smash{\overbrace{\martIncr_{k+1}}^{X_{k+1}-\weights_*}}.
\end{align*}
This implies we do not weight our \(X_k\) equally, but rather use an
exponential decay giving the most recent data the most weight.

Notice how we can reobtain the rates from Theorem~\ref{thm: distance SGD vs GD}
by throwing this entire exponential decay away
\begin{align*}
	\E\|\Weights_n-\weights_n\|^2
	&= \E\|\sum_{k=0}^{n-1}\lr(1-\lr)^{n-1-k}\martIncr_{k+1}\|^2\\
	&= \sum_{k=0}^{n-1}\lr^2\underbrace{(1-\lr)^{2(n-1-k)}}_{\le1}
	\underbrace{\E\|\martIncr_{k+1}\|^2}_{\le K}\\
	&\le \lr^2 n K \le \lr TK.
\end{align*}
We can do the same thing with the general case (\ref{eq: unrolled SGD weights
(general quadratic loss case)}) if we take \(\lr<2/\ubound\) for \(|H|\le\ubound\).

Now this fact is making me quite anxious about the claim that batch learning is
worse than SGD in general. But we also have to keep in mind how special this
case is. In (\ref{eq: getting rid of w_0 first step}) we can get rid of the
initial bias \(\weights_0\) completely. This is only possible because our
condition number \(\condition\) is one which allows one step convergence of GD.

\subsubsection{The Effect of the Condition Number}

Now by diagonalizing \(H\) and using the fact that
\begin{align*}
	&\prod_{k=0}^{n-1} (1-\lr_k V\diag[\hesseEV_1, \dots, \hesseEV_\dimension]V^T)\\
	&= V \diag\left[
		\prod_{k=0}^{n-1}(1-\lr_k\hesseEV_1),
		\dots, \prod_{k=0}^{n-1}(1-\lr_k\hesseEV_\dimension)
	\right]V^T
\end{align*}
we can consider all the eigenspaces separately again and it is also quite
obvious that again only the largest and smallest eigenvalue will matter.
With the usual conditional independence argument we get

\section{Expected Square Error Analysis}

\subsection{Strongly Convex Case}

While we are not going to tether ourselves to Gradient Decent \(\weights_n\) in
this section, we are still going to split off the noise
\begin{align*}
	\|\Weights_{n+1} - \weights_*\|^2
	&= \|\Weights_n -\weights_* - \lr_n\nabla\Loss(\weights_n) +\lr_n \martIncr_{n+1}\|^2\\
	&= \begin{aligned}[t]
		&\|\Weights_n - \weights_* - \lr_n\nabla\Loss(\Weights_n)\|^2\\
		&+ 2\langle \Weights_n - \weights_* - \lr_n\nabla\Loss(\Weights_n), \lr_n \martIncr_{n+1}\rangle\\
		&+ \lr_n^2 \|\martIncr_{n+1}\|^2
	\end{aligned}
\end{align*}
The scalar product will disappear once we apply expectation due to conditional
independence again. For the first part we are going to do the same as in the
classical case in Theorem~\ref{thm: gd strong convexity convergence rate}:
\begin{align*}
	&\|\Weights_n - \weights_* - \lr_n\nabla\Loss(\Weights_n)\|^2\\
	&= \|\Weights_n - \weights_*\|^2
	- 2\lr_n\underbrace{\langle\nabla\Loss(\Weights_n), \Weights_n-\weights_*\rangle}_{
		\xge{\text{Lem.~\ref{lem: bermanDiv lower bound (strongly convex)}}}
	 	\tfrac{\ubound\lbound}{\ubound+\lbound}\|\Weights_n - \minimum\|^2
		\mathrlap{+ \tfrac{1}{\ubound+\lbound}\|\nabla\Loss(\Weights_n)\|^2}
	}
	+ \lr_n^2 \|\nabla\Loss(\Weights_n)\|^2\\
	&\le \left(1-2\lr_n\tfrac{\ubound\lbound}{\ubound+\lbound}\right)
	\|\Weights_n - \weights_*\|^2
	+ \lr_n(\lr_n - \tfrac{2}{\ubound+\lbound})
	\|\nabla\Loss(\Weights_n)\|^2
\end{align*}
For \(\lr_n\le\tfrac{2}{\ubound+\lbound}\) we can drop the gradient again to
obtain
\begin{align*}
	\E\|\Weights_{n+1} - \weights_*\|^2
	\le \left(1-2\lr_n\tfrac{\ubound\lbound}{\ubound+\lbound}\right)
	\E\|\Weights_n - \weights_*\|^2 + \lr_n^2 \E\|\martIncr_{n+1}\|^2.
\end{align*}
In the classical case we just wanted to select \(\lr_n\) as large as possible,
i.e. \(\lr_n=\tfrac{2}{\ubound+\lbound}\). But in the stochastic setting there
are adverse effect of large learning rates due to noise. Now as our upper bound
is a convex parabola in \(\lr_n\) we can just use the first order condition
\begin{align*}
	\frac{d}{d\lr_n}
	= -2\tfrac{\ubound\lbound}{\ubound+\lbound}
	\E\|\Weights_n - \weights_*\|^2 + 2\lr_n \E\|\martIncr_{n+1}\|^2
	\xeq{!} 0
\end{align*}
to find the best upper bound
\begin{align*}
	\lr_n
	= \min\left\{\frac{
		\tfrac{\ubound\lbound}{\ubound+\lbound}\E\|\Weights_n - \weights_*\|^2
	}{\E\|\martIncr_{n+1}\|^2},
	\frac{2}{\ubound+\lbound}
	\right\}.
\end{align*}
In particular we still want to ``max out our learing rate'', i.e. select
\(\lr_n=\tfrac{2}{\ubound+\lbound}\) if
\begin{align*}
	2\E\|\martIncr_{n+1}\|^2 \le \ubound\lbound \E\|\Weights_n - \weights_*\|^2,
\end{align*}
in other words: the stochastic variance only becomes a problem once we are
close to the minimum. Before that we basically want to treat our problem like
the classical one. This results in two training phases, the \emph{transient}
phase in which we are still far from the minimum, our learning rate is constant
and we get a linear convergence rate
\begin{align*}
	\E\|\Weights_{n+1} - \weights_*\|^2
	&\le \left(1-2\tfrac{2}{\ubound+\lbound}\tfrac{\ubound\lbound}{\ubound+\lbound}\right)
	\E\|\Weights_n - \weights_*\|^2 + \left(\tfrac{2}{\ubound+\lbound}\right)^2
	\underbrace{\E\|\martIncr_{n+1}\|^2}_{
		\le\tfrac{\ubound\lbound}{2} \E\|\Weights_n - \weights_*\|^2
	}\\
	&\le \underbrace{
		\left(1-\tfrac{2\ubound\lbound}{(\ubound+\lbound)^2}\right)
	}_{
		=1-\frac{2}{(1+\condition^{-1})(1+\condition)}
	}
	\E\|\Weights_n - \weights_*\|^2.
\end{align*}
In the \emph{asymptotic} phase on the other hand, we have to reduce our learning
rate with the distance to the minimum


\subsection{General Convex Case}

\textcite{nemirovskiRobustStochasticApproximation2009}


\subsection{Averaging}

cf. \cite{bachNonstronglyconvexSmoothStochastic2013}

\section{SDE View}

\textcite{simsekliTailIndexAnalysisStochastic2019}


\section{Heuristics}

\subsection{Adagrad}

\subsection{Adadelta}

\subsection{RMSProp}

%%%%%%%%%%%%%%%%%%%%%%%%%%%%%%%%%%%

\endinput
