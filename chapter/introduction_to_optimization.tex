% !TEX root = ../Masterthesis.tex
\newcommand{\dist}{\mathcal{D}}
\newcommand{\E}{\mathbb{E}}

\chapter{Introduction to Optimization}


Let \(f_\theta(X)\) be a (parametrized) prediction for \(Y\), where \(Z=(X,Y)\)
is drawn from a distribution \(\dist\) of real world examples. The prediction
error of our parameter \(\theta\) is a loss function \(l(\theta, X, Y)\), e.g. 
%
\begin{align*}
	l(\theta, x, y) := (f_\theta(x) - y)^2.
\end{align*}
%
And we \emph{want} to minimize the theoretical loss over all examples (risk)
%
\begin{align*}
	L(\theta) := \E_{X,Y\sim \dist} [l(\theta, X, Y)].
\end{align*}

\section{Zero Order Optimization}

Let us set aside the stochasticity of the problem for a moment and assume we
could evaluate \(L\) at any point \(\theta\). In other words: we have a "zero
order oracle" which can tell us \(L(\theta)\) for any \(\theta\) for a fixed
computational cost.

If we assume an infinite number of possible inputs \(\theta\), it does not matter
how much and where we sample \(L\) there is always a different \(\theta\) where
\(L\) might be arbitrarily lower. This notion is formalized in various "No-Free-Lunch
Theorems". We must therefore start making assumptions about \(L\) to make this
problem tractable.

\subsection{Grid Search}


While Lipschitz continuity 
%
\begin{align*}
	| L(\theta_1) - L(\theta_2) | < K_0 | \theta_1 - \theta_2 |
\end{align*}
%
is sufficient to find an \(\epsilon\)-solution \(\hat{\theta}\) inside a unit
(hyper-)cube
%
\begin{align*}
	L(\hat{\theta}) \le \inf_{\theta: 0\le\theta_i\le 1} L(\theta) + \epsilon
\end{align*}
%
in finite time, it still requires a full \emph{grid search} with time complexity
%
\begin{align*}
	\left(\left\lfloor \frac{K_0}{2\epsilon}\right\rfloor + 1\right)^n
\end{align*}
%
where \(n\) is the dimension of \(\theta\). The fact that grid search is
sufficient can be seen by tiling the space into a grid and asking the oracle
for the value of the loss function at each center. If the minimum is placed in
some tile, the Lipschitz continuity forces the center of the tile to be \(\epsilon\)
close to it \parencite[cf.][p. 11]{nesterovLecturesConvexOptimization2018}.

The fact that this is necessary can easily be seen by imagining a "resisting
oracle" which places the minimum in just the last grid element where we look
\parencite[cf.][p. 13]{nesterovLecturesConvexOptimization2018}. The exponential
increase of complexity in the dimension makes the problem intractable for even
mild dimensions and precisions (e.g. \(n=11\), \(\epsilon=0.01\), \(K_0=2\)
implies millions of years of computation). So we have to make further
assumptions. 

\fxnote{simulated annealing? evolutionary algorithms?}{\subsection{Heuristics}}


%%%%%%%%%%%%%%%%%%%%%%%%%%%%%%%%%%%

\endinput
