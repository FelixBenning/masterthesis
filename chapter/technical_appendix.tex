
% !TEX root = ../Masterthesis.tex

\chapter{Technical Proofs}

\section{Convex Analysis}

\begin{lemma}\label{Appdx-lem: lipschitz and bounded derivative}
	If \(f\) is differentiable, then the derivative \(\nabla f\) is
	bounded (w.r.t. the operator norm) by constant \(\lipConst\) if and only if the function
	\(f\) is \(\lipConst\)-Lipschitz continuous.
\end{lemma}
\begin{proof}
	Lipschitz continuity is implied by the mean value theorem
	\begin{align*}
		\|f(x_1) - f(x_0)\|
		&\le \|\nabla f(x_0 + \xi(x_1-x_0))\| \|x_1- x_0\|\\
		&\le \lipConst \|x_1-x_0\|.
	\end{align*}
	%
	The opposite direction is implied by
	%
	\begin{align*}
		\|\nabla f(x_0)\|
		&\equiv \sup_{v} \frac{\|\nabla f(x_0)v\|}{\|v\|}
		= \sup_{v} \lim_{v\to 0}\frac{\|\nabla f(x_0)v\|}{\|v\|}\\
		&\lxle{\Delta}\sup_{v} \lim_{v\to 0}
		(
			\underbrace{
				\tfrac{\|f(x_0 + v) - f(x_0) \|}{\|v\|}
			}_{
				\le \lipConst 
			}
			+ \underbrace{
				\tfrac{\|\nabla f(x_0)v + f(x_0) - f(x_0 +v)\|}{\|v\|}
			}_{
				\to 0 \text{ (derivative definition)}
			}
		)
	\end{align*}
	%
	where we have used the scalability of the norm to multiply \(v\) with a
	decreasing factor both in the numerator and denominator in order to introduce
	the limit.
\end{proof}

\begin{lemma}
	\label{Appdx-lem: Lipschitz Gradient implies taylor inequality}
	If \(\nabla f\) is \(\ubound\)-Lipschitz continuous, then
	\begin{align*}
		|f(y) - f(x) - \langle \nabla f(x), y-x\rangle | \le \tfrac{\ubound}2 \|y-x\|^2
	\end{align*}
	If \(f\) is convex, then the opposite direction is also true.
\end{lemma}
\begin{proof}
	The first direction is taken from \textcite[Lemma
	1.2.3]{nesterovLecturesConvexOptimization2018}.
 \begin{align*}
		f(y) = f(x) + \int_0^1\langle\nabla f(x+\tau(y-x)), y-x \rangle d\tau
	\end{align*}
	implies (using the  Cauchy-Schwarz inequality)
	\begin{align*}
		&| f(y) - f(x) - \langle \nabla f(x), y-x\rangle | \\
		&\le \int_0^1 | \langle\nabla f(x+\tau(y-x))-\nabla f(x), y-x\rangle | d\tau \\
		&\lxle{\text{C.S.}}
		\int_0^1 \|\langle\nabla f(x+\tau(y-x))-\nabla f(x)\| \cdot \|y-x\| d\tau\\
		&\le \int_0^1 \ubound \|\tau(y-x)\|\cdot\|y-x\| d\tau
		= \tfrac{\ubound}2 \|y-x\|^2.
		\qedhere
	\end{align*}
	The opposite direction is taken from \textcite[Lemma
	2.1.5]{nesterovLecturesConvexOptimization2018}.
	As we have only used
	\begin{align*}
		0\xle{\text{Convexity}} f(y) - f(x) - \langle \nabla f(x), y-x\rangle
		\le \tfrac{\ubound}2 \|y-x\|^2
	\end{align*}	
	in Lemma~\ref{lem: bermanDiv lower bound} (and Lemma~\ref{lem: smallest upper
	bound} which was used in the proof), we know that equation (\ref{eq:
	bergmanDiv lower bound b}) follows from it and after applying Cauchy-Schwarz
	to this equation
	\begin{align*}
		\tfrac{1}\ubound \|\nabla f(x)-\nabla f(y)\|^2
		&\le \langle \nabla f(x) - \nabla f(y), x-y\rangle \\
		&\lxle{\text{C.S.}} \|\nabla f(x) - \nabla f(y)\| \|x-y\|,
	\end{align*}
	we only have to divide both sides by \(\tfrac{1}\ubound \|\nabla f(x)-\nabla
	f(y)\|\) to get \(\ubound\)-Lipschitz continuity back.
\end{proof}

\section{Momentum Convergence}

\begin{lemma}[\cite{qianMomentumTermGradient1999}]
	Let
	\begin{align*}
		\sigma_{1/2}
		= \tfrac12 \left(
			1+\beta-\alpha\lambda \pm \sqrt{(1+\beta-\alpha\lambda)^2 - 4\beta}
		\right)
	\end{align*}
	then \(\max\{|\sigma_1|,|\sigma_2|\}<1\) if and only if
	\begin{align*}
		0<\alpha\lambda < 2(1+\beta) \qquad \text{and} \qquad |\beta|<1
	\end{align*}
\end{lemma}
\begin{proof}
	Define
	\begin{align*}
		\Delta(\beta)
		&:= (1+\beta-\alpha\lambda)^2 - 4\beta\\
		&= \beta^2 - 2(1+\alpha\lambda)\beta + (1-\alpha\lambda)^2
	\end{align*}
	then \(\sigma_{1/2}\) is complex iff \(\Delta(\beta)<0\). Since it is a
	convex parabola this implies that \(\beta\) needs to be between the roots
	\begin{align*}
		\beta_{1/2}
		= (1+\alpha\lambda) \pm 
		\underbrace{
			\sqrt{(1+\alpha\lambda)^2 - (1-\alpha\lambda)^2}
		}_{=\sqrt{4\alpha\lambda}=\mathrlap{2\sqrt{\alpha\lambda}}}
		= (1\pm \sqrt{\alpha\lambda})^2	
	\end{align*}
	of \(\Delta\). Assuming those roots are real which requires \(\alpha\lambda>0\).
	But if \(\alpha\lambda\le0\) then we would have
	\begin{align*}
		\sigma_1
		= \tfrac12 \Big(
			\underbrace{1+\beta+|\alpha\lambda|}_{\ge 1+\beta}
			+ \sqrt{\underbrace{(1+\beta+|\alpha\lambda|)^2 - 4\beta}_{\smash{\ge (1-\beta)^2}}}
		\Big)
		\ge \frac{1+\beta + |1-\beta|}2 \ge 1
	\end{align*}
	which means that \(\max\{|\sigma_1|,|\sigma_2|\}<1\) implies \(\alpha\lambda>0\).
	The roots \(\beta_{1/2}\) are therefore real no matter the direction we
	wish to prove.
	\begin{description}[wide, labelindent=0pt]
	\item[Complex Case:]
		\(\sigma_{1/2}\) are complex iff	
		\begin{align}\label{eq: complex case}
			0 < (1-\sqrt{\alpha\lambda})^2 < \beta < (1+\sqrt{\alpha\lambda})^2.
		\end{align}
		In that case we have
		\begin{align*}
			|\sigma_1| = |\sigma_2|
			&= \sqrt{|\Re(\sigma_1)|^2 + |\Im(\sigma_1)|^2}\\
			&= \tfrac12 \sqrt{(1+\beta-\alpha\lambda)^2 + 4\beta - (1+\beta-\alpha\lambda)^2}
			= \sqrt{\beta}.
		\end{align*}
		So in this complex case the condition
		\(\beta<1\) is necessary and sufficient for \(\max\{|\sigma_1|,|\sigma_2|\}<1\).
		And since we have seen that \(\alpha\lambda>0\) is necessary and \(0<\beta\)
		from (\ref{eq: complex case}) implies the condition \(|\beta|<1\), we only
		need to show that \(\alpha\lambda <2(1+\beta)\) is necessary in the complex
		case to have this case covered. Using \((\sqrt{\alpha\lambda}-1)^2 < \beta\)
		from (\ref{eq: complex case}) and \(ab \le a^2 + b^2\) we get
		\begin{align*}
			\sqrt{\alpha\lambda} - 1
			< \beta \implies \alpha\lambda < (1+\sqrt{\beta})^2
			\le 2(1+\beta)
		\end{align*}
		and therefore have covered this case.
		\item[Real Case:] Since we have \(\sigma_2 \le \sigma_1\),
		\(\max\{|\sigma_1|,|\sigma_2|\}<1\) is equivalent to
		\begin{align*}
			-1 < \sigma_2 \qquad \text{and} \qquad \sigma_1 < 1
		\end{align*}
		By subtracting the part before the "\(\pm\)" from the equation \(\sigma_1<1\)
		we get
		\begin{align}\label{eq: sigma1 condition}
			0\xle{\text{Real Case}} \sqrt{(1+\beta-\alpha\lambda)^2 -4\beta} < 1-\beta +\alpha\lambda.
		\end{align}
		And since \((1+\sqrt{\alpha\lambda})^2 \le \beta\) leads to a contradiction:
		\begin{align*}
			0 \le 1-\beta+\alpha\lambda \le 1-(1+\sqrt{\alpha\lambda})^2 +\alpha\lambda
			= -2\sqrt{\alpha\lambda} < 0
		\end{align*}
		the real case is restricted to the case \(\beta < (1-\sqrt{\alpha\lambda})^2\)
		as we are otherwise in the complex case (\ref{eq: complex case}).
		This means that due to
		\begin{align*}
			1-\beta+\alpha\lambda
			> 1 - (1-\sqrt{\alpha\lambda})^2 + \alpha\lambda
			= 2\sqrt{\alpha\lambda}>0
		\end{align*}
		the inequality (\ref{eq: sigma1 condition}) is (in the real case) equivalent to
		\begin{align*}
			&(1+\beta-\alpha\lambda)^2 - 4\beta < (1-\beta+\alpha\lambda)^2\\
			&\iff \cancel{1^2} + 2(\beta-\alpha\lambda) + 
			\cancel{(\beta-\alpha\lambda)^2} - 4\beta
			< \cancel{1^2} - 2(\beta-\alpha\lambda) + 
			\cancel{(\beta-\alpha\lambda)^2}\\
			&\iff 0 < 4\alpha\beta.
		\end{align*}
		In other words it is a given for positive learning rates in the convex	
		case which is not surprising if one remembers the proof from gradient
		decent. The real issue is not overshooting \(-1\) with large learning
		rates. So multiplying \(-1<\sigma_2\) by two, adding 2 and moving the
		root to the other side we get
		\begin{align}\label{eq: from -1<sigma_2 followed}
			0 \xle{\text{Real Case}} \sqrt{(1+\beta-\alpha\lambda)^2 - 4\beta}
			< 3+\beta-\alpha\lambda
		\end{align}
		Now if we assume \(\alpha\lambda < 3+\beta\), then this condition is
		equivalent to 
		\begin{align}
			\nonumber
			&(1+\beta-\alpha\lambda)^2 - 4\beta < (3+\beta-\alpha\lambda)\\
			\nonumber
			&\iff 1^2 + 2(\beta-\alpha\lambda) + \cancel{(\beta-\alpha\lambda)^2} - 4\beta
			< 3^2 + 6(\beta-\alpha\lambda) + \cancel{(\beta-\alpha\lambda)^2}\\
			\label{eq: momentum upper bound derivation}
			&\iff 4\alpha\lambda < 8 + 8\beta
			\iff \alpha\lambda < 2(1+\beta)
		\end{align}
		which is the important requirement on the learning rate we are looking for.
		Now for \(\beta<1\) this requirement is actually stronger than
		\(\alpha\lambda <3+\beta\) which means that we have proven that our
		requirements are sufficient for \(\max\{|\sigma_1|,|\sigma_2|\}<1\).
		What is left to show is that \(|\beta|<1\) is also necessary in the real
		case. Now from (\ref{eq: momentum upper bound derivation}) we immediately
		get
		\begin{align*}
			0 \le \tfrac{\alpha\lambda}2 < 1+\beta
		\end{align*}
		which implies \(-1 < \beta\). At first glance it might look like
		\(\beta<(1-\sqrt{\alpha\lambda})^2\) is already sufficient for the upper
		bound, but for \(\alpha\lambda>4\) this bound is greater than one again.
		So let us assume \(\alpha\lambda>2\) and \(\beta>0\), then
		\(\beta<(1-\sqrt{\alpha\lambda})^2\) implies
		\begin{align*}
			\sqrt{\beta} < \sqrt{\alpha\beta} -1
			&\implies \smash{\overbrace{(1+\sqrt{\beta})^2}^{1+2\sqrt{\beta} + \beta}}
			< \alpha\lambda
			\stackrel{(\ref{eq: from -1<sigma_2 followed})}{<} 3+\beta \\
			&\implies -2 +2\sqrt{\beta} < 0
			\implies \sqrt{\beta} < 1
			\qedhere
		\end{align*}
 \end{description}
\end{proof}

%%%%%%%%%%%%%%%%%%%%%%%%%%%%%%%%%%%

\endinput