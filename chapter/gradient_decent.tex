% !TEX root = ../Masterthesis.tex

\newcommand{\identity}{\mathbb{I}}
\newcommand{\lbound}{\mu_l}
\newcommand{\ubound}{\mu_u}
\newcommand{\diag}{\text{diag}}
\newcommand{\rate}{\text{rate}}
\newcommand{\reals}{\mathbb{R}}
\newcommand{\firstOrderMethod}{\mathcal{M}}

\chapter{Gradient Decent (GD)}

A step up from zero order methods (such as grid search) are first order methods
using the first derivative (the gradient). This requires a "first order oracle"
which can provide us with \(L(\theta)\) and \(\nabla L(\theta)\) at any point
\(\theta\).

One immediately obvious way to utilize this information is to
incrementally move in the direction of steepest decent, i.e.
%
\begin{align*}
	\theta_{n+1} = \theta_n - \eta\nabla L(\theta_n).
\end{align*}
%
A useful way to look at this equation is to notice that it is the
discretization of an ordinary differential equation (ODE)
%
\begin{align*}
	\dot{\theta}_n \approx \frac{\theta_{n+1} - \theta_n}{\eta}
	= - \nabla L(\theta_n).
\end{align*}
%
Here the "learning rate" \(\eta\) is the time delta between discretizations
\(t_n\) and \(t_{n+1}\). This of course implies \(t_n = n\eta\). And for
\(\eta\to\infty\), we arrive at the ODE
%
\begin{align}\label{eq: velocity is gradient}
	\dot{\theta}(t) = -\nabla L(\theta(t)).
\end{align}
%
If you are familiar with Lyapunov functions you will not be surprised by the next
argument:
%
\begin{align}\label{eq: gradient integral}
	L(\theta(t_1)) - L(\theta(t_0))
	&= \int_{t_0}^{t_1} \nabla L(\theta(s)) \cdot \dot{\theta}(s) ds
	= \int_{t_0}^{t_1} -\|\nabla L(\theta(s))\|^2 ds
	\le 0
\end{align}
%
immediately implies
\begin{align*}
	L(\theta(t_0)) \ge L(\theta(t_1)) \ge \dots \ge \inf_\theta L(\theta) \ge 0
\end{align*}
which implies convergence. But not necessarily a convergent \(\theta(t)\) and
not necessarily convergence to \(\inf_\theta L(\theta)\).

\section{Visualizing the 2nd Taylor Approximation}\label{sec: visualize gd}

To build some intuition what leads to convergence of \(\theta(t)\) let us
consider more simple cases. If we assume our Hesse matrix \(\nabla^2
L(\theta)\) exists and is positive definite (all eigenvalues positive), then for
the second order Taylor approximation
%
\begin{align*}
	L(\theta+v) \approx T_2L(\theta+v)
	= L(\theta) + v^T \nabla L(\theta) + \tfrac12 v^T \nabla^2 L(\theta) v
\end{align*}
%
we can find the minimum 
\begin{align*}
	\hat{v} = -(\nabla^2 L(\theta))^{-1}\nabla L(\theta)
\end{align*}
by setting the first derivative to zero
%
\begin{align*}
	\nabla T_2L(\theta+v) = \nabla L(\theta) + \nabla^2L(\theta) v \xeq{!} 0.
\end{align*}
%
This minimum allows us not only to rewrite the taylor derivative
%
\begin{align*}
	\nabla T_2L(\theta+v) = \nabla^2 L(\theta)(v-\hat{v}),
\end{align*}
%
but also the original taylor approximation
%
\begin{align*}
	T_2L(\theta+v)
	&= L(\theta) - v^T \nabla^2 L(\theta) \hat{v} + \tfrac12 v^T \nabla^2 L(\theta) v \\
	&= \underbrace{L(\theta) - \tfrac12 \hat{v} \nabla^2 L(\theta) \hat{v}}_{=: c(\theta) \text{ (const.)}}
	+ \tfrac12 (v-\hat{v})^T \nabla^2 L(\theta)(v-\hat{v}).
\end{align*}
%
To get absolute instead of relative coordinates to \(\theta\) we set
%
\begin{align*}
	\hat{\theta} := \theta + \hat{v} = \theta -(\nabla^2 L(\theta))^{-1}\nabla L(\theta),
\end{align*}
%
and obtain the notion of a paraboloid centered around \(\hat{\theta}\)
%
\begin{align}\label{paraboloid approximation of L}
	L(y) = \tfrac12 (y- \hat{\theta}) \nabla^2 L(\theta) (y-\hat{\theta}) + c(\theta) + o(\|y-\theta\|^2)
\end{align}
%
\begin{wrapfigure}{O}{0.65\textwidth}
	\centering
	\def\svgwidth{0.65\textwidth}
	\input{media/contour.pdf_tex}
	\caption{Assuming \(\hat{\theta}=0\), \(\lambda_1=1, \lambda_2=2\), \(v_1=(\sin(1), \cos(1))\)}
	\label{fig: 2d paraboloid}
\end{wrapfigure}
%
To fully appreciate Figure~\ref{fig: 2d paraboloid}, we now only need to realize
that the diagonizability of \(\nabla^2 L(\theta)\)
%
\begin{align}\label{eq: diagnalization of the Hesse matrix}
	V \nabla^2 L(\theta) V^T
	= \diag(\lambda_1,\dots,\lambda_d), \qquad V=(v_1,\dots, v_d)
\end{align}
%
implies that once we have selected the center \(\hat{\theta}\) and the direction of
one eigenspace in two dimensions, the other eigenspace has to be
orthogonal to the first one which fully determines its direction at this point. 

Before we move on I want to briefly mention that we only really needed the
positive definiteness of \(\nabla^2 L(\theta)\) to make it invertible, so that
\(\hat{v}\) is well defined. If it was not positive definite but still invertible,
then \(\hat{v}\) would not be a minimum but all the other arguments would still
hold.
In that case the eigenvalues might be negative as well as positive, which would
represent a saddle point, or all negative which would represent a maximum.

Using the representation (\ref{paraboloid approximation of L}) of \(L\) we
can write the gradient at \(\theta\) as
%
\begin{align*}
	\nabla L(\theta)
	=  \nabla^2 L(\theta)(\theta-\hat{\theta})
	\ (= -\nabla^2 L(\theta)\hat{v})
\end{align*}
%
But note that \(\hat{\theta}\) depends on \(\theta\), so we need to index both
by \(n\) to rewrite gradient decent
%
\begin{align*}
	\theta_{n+1} &= \theta_n - \eta\nabla L(\theta_n)\\
	&= \theta_n - \eta\nabla^2 L(\theta_n)(\theta_n - \hat{\theta}_n).
\end{align*}
%
Subtracting \(\hat{\theta}_n\) from both sides we obtain the following
transformation 
%
\begin{align*}
	\theta_{n+1} - \hat{\theta}_n
	&= (\identity - \eta\nabla^2 L(\theta_n) ) (\theta_n - \hat{\theta}_n).
\end{align*}
%
Taking a closer look at this transformation matrix we can use (\ref{eq:
diagnalization of the Hesse matrix}) to see
%
\begin{align*}
	\identity - \eta\nabla^2 L(\theta_n)
	&= V(\identity - \eta\cdot\diag(\lambda_1,\dots,\lambda_d) )V^T \\
	&= V\cdot\diag(1-\eta\lambda_1, \dots,1-\eta\lambda_d)V^T.
\end{align*}
%
Now if we assume like \textcite{gohWhyMomentumReally2017}, that the second
taylor approximation is accurate and thus that \(\nabla^2 L(\theta)=H\) is a
constant, then \(\hat{\theta}_n = \theta^*\) is the real minimum and we get
%
\begin{align}
	\theta_n - \theta^*
	= V\cdot\diag[(1-\eta\lambda_1)^n,\dots,(1-\eta\lambda_d)^n] V^T (\theta_0 - \theta^*)
\end{align}
%
by induction. Decomposing the difference into the eigenspaces of \(H\) we can 
see that each component scales exponentially on its own 
%
\begin{align*}
	\langle \theta_n -\theta^*, v_i\rangle
	= (1-\eta\lambda_i)^n \langle \theta_0 - \theta^*, v_i\rangle
\end{align*}
%
This is beautifully illustrated with interactive graphs in
\citetitle{gohWhyMomentumReally2017} by \citeauthor{gohWhyMomentumReally2017}.

\subsection{Negative eigenvalues}

Now if \(\lambda_i<0\), then \(1-h\lambda_i\) would be greater
than one which would repel this component \(\langle \theta_0 - \theta^*,
v_i\rangle\) away from \(\theta^*\). This is a good thing, since \(\theta^*\)
is a maximum in this component. This means we will walk down local minima and
saddle points no matter the learning rate assuming  this component  was not
zero to begin with. In case of a maximum this would mean that we would start
right on top of the maximum, in case of a saddle point it implies starting on
the rim such that one can not roll down either side.

But since we are ultimately interested in stochastic loss functions, the
stochasticity would push us off these narrow equilibria so being right on
top of them is of little concern. But being close to zero in such a component
still means slow movement away, since we are multiplying by it. This is a
common explanation for the observation of temporary plateaus in machine learning
which "suddenly fall off" once the exponential factor ramps up and causes a
sharp drop in the loss (Figure~\ref{fig: visualize saddlepoint gd}).
%
\begin{figure}[h]
	\centering
	\def\svgwidth{1\textwidth}
	\input{media/visualize_gd.pdf_tex}
	\caption{Start=\(0.001v_1+4v_2\), \(\lambda_1=-1, \lambda_2=2\), learning rate\(=0.8\)}
	\label{fig: visualize saddlepoint gd}
\end{figure}

\subsection{Assuming Convexity}

Let us consider strictly positive eigenvalues now and assume that our
eigenvalues are already sorted
%
\begin{align}
	0 < \lambda_1 \le \dots \le \lambda_d.
\end{align}
%
Then for positive learning rates all exponentiation bases are smaller than one
%
\begin{align*}
	1-\eta\lambda_d \le \dots \le 1-\eta\lambda_1 \le 1.
\end{align*}
%
But to ensure convergence we also need that all of them are larger than \(-1\).
This leads us to the condition
\begin{align}\label{eq: learning rate restriction (eigenvalue)}
	0< \eta < 2/\lambda_d
\end{align}
%
Selecting \(\eta = 1/\lambda_d\) reduces the exponentiation base ("rate") of the
corresponding eigenspace to zero ensuring convergence in one step.
But if we want to maximize the convergence rate of \(\theta_n\), this is not
the best selection.

We can reduce the rates of the other eigenspaces if we
increase the learning rate further, getting them closer to zero. At this point
we are of course overshooting zero with the largest learning rate, so we only
continue with this until we get closer to \(-1\) with the largest eigenvalue
than we are to \(1\) with the smallest:
%
\begin{align*}
	\rate(\eta)=\max_{i} |1-\eta\lambda_i| = \max\{|1-\eta\lambda_1|, |1-\eta\lambda_d|\}.
\end{align*}
%
This is minimized when
%
\begin{align*}
	1-\eta\lambda_1 = \eta\lambda_d -1,
\end{align*}
%
implying
%
\begin{align*}
	\eta^* = \frac{2}{\lambda_1 + \lambda_d}.
\end{align*}
%
If \(\lambda_1\) is much smaller than \(\lambda_d\), this leaves \(\eta\)
at the upper end of the interval in (\ref{eq: learning rate restriction
(eigenvalue)}). And the optimal convergence rate
%
\begin{align*}
	\rate(\eta^*)
	% &= 1-\eta^* \lambda_1
	= 1 - \frac{2}{1+\kappa}
	\qquad \kappa:=\lambda_d/\lambda_1 \ge 1
\end{align*}
%
becomes close to one if the condition number \(\kappa\) is large.
If all the eigenvalues are the same on the other hand, the condition number
becomes one and the rate is zero, implying instant convergence. This is not
surprising if we recall our visualization in Figure~\ref{fig: 2d paraboloid}.
When the eigenvalues are the same, the contours are concentric circles and the
gradient points right at the center.

\fxnote{strong convexity and lipshitz continuity of gradient as lower and
upper bound, sketch convergence proof using that instead of eigenvalues}

\section{Lipschitz Gradient}

Let us try to get rid of the assumptions we made in \ref{sec: visualize gd}.
The most egregious assumption was a constant second derivative. We used this
constant second derivative to determine the (constant) condition number of
the (constant) eigenvalues. Since we used only the smallest and largest
eigenvalue, it is natural to guess that simply bounding the second derivative
from above and below should be sufficient:
%
\begin{align*}
	\lbound \identity \precsim \nabla^2 L(\theta) \precsim \ubound \identity.
\end{align*}
%
Here \(A \precsim B\) should be read as \(0\precsim B-A\) which we define to mean
\(B-A\) is positive definite. Using the orthonormal basis of eigenvectors of
\(\nabla^2 L(\theta)\) to represent \(x\) in the definition of the operator norm
%
\begin{align*}
	\|A\| := \sup_{\|x\| =1} \|Ax\|
	\left(= \sup_{\|x\| =1} \sqrt{\langle Ax, Ax\rangle}\right)
\end{align*}
%
it becomes immediately clear that the operator norm is equal to the largest
absolute eigenvalue. This implies that for positive eigenvalues 
%
\begin{align*}
	\nabla^2 L(\theta) \precsim \ubound\identity
	\iff \|\nabla^2 L(\theta)\|\le \ubound,
\end{align*}
%
and the following lemma allows us to get rid of the existence of the second
derivative for the upper bound entirely
%
\begin{lemma}\label{lem: lipschitz and bounded derivative}
	If \(f\) is differentiable, then the derivative \(\nabla f\) is
	bounded (w.r.t. the operator norm) by constant \(K\) if and only if the function
	\(f\) is Lipschitz continuous with Lipschitz constant \(K\).
\end{lemma}
\begin{proof}
	See Appendix \ref{Appdx-lem: lipschitz and bounded derivative}.
\end{proof}
%
\noindent
Now we have to ask ourselves whether Lipschitz continuity of the derivative
\(\nabla L\)
%
\begin{align*}
	\| \nabla L(\theta) - \nabla L(\tilde{\theta})\|
	\le \ubound \|\theta-\tilde{\theta}\|
\end{align*}
%
is a reasonable assumption. There are two somewhat convincing reasons why there
is probably not much more space for generalization.

First, we are using our derivative to define an ODE (\ref{eq: velocity is
gradient}) and it is very common to use Lipschitz continuity to argue for
existence, uniqueness and stability of ODEs. While these properties might not
be needed for effective optimization, working without them will likely be very
cumbersome.

Second, we need at least uniform continuity of \(\nabla L\) so that the inequality
%
\begin{align}\label{eq: bounded gradient integral}
	\int_{t_0}^\infty \|\nabla L(\theta(s))\|^2 ds
	&\le L(\theta(t_0)) - \liminf_{t\to\infty} L(\theta(t)) \\
	&\le L(\theta(t_0)) - \inf_{\theta} L(\theta) < \infty \nonumber
\end{align}
%
derived from (\ref{eq: gradient integral}) is sufficient for a convergent
\(\|\nabla L(\theta(t))\|\). But that still does not ensure that \(\theta(t)\)
converges, only that its derivative \(\dot{\theta}(t) = -\nabla L(\theta(t))\)
converges. The logarithm is an obvious example where this goes wrong.

Before we deal with that problem, let us make sure that we can get convergence
of the discretized version of gradient decent \(\theta_{n+1}=\theta_n-\eta\nabla
L(\theta_n)\) too. 
%
\begin{lemma}[\citeauthor{nesterovLecturesConvexOptimization2018}]
	\label{lem: Lipschitz Gradient implies taylor inequality}
	Assume \(\nabla f\) is Lipschitz continuous with Lipschitz constant \(K\),
	then we have
	\begin{align*}
		| f(y) - f(x) - \langle \nabla f(x), y-x\rangle | \le \tfrac{K}2 \|y-x\|^2
	\end{align*}
\end{lemma}
\begin{proof}
	See Appendix \ref{Appdx-lem: Lipschitz Gradient implies taylor inequality}.
\end{proof}
%
\noindent
Using the Lipschitz continuity of the gradient and Lemma~\ref{lem: Lipschitz
Gradient implies taylor inequality} we get
%
\begin{align}
	L(\theta_{n+1})
	&\le L(\theta_n) 
	+ \langle\nabla L(\theta_n),\theta_{n+1} - \theta_n\rangle
	+ \tfrac{\ubound}{2} \| \theta_{n+1} - \theta_n\|^2 
	\label{bound increment}\\
	&= L(\theta_n)
	+ \langle\nabla L(\theta_n), -\eta\nabla L(\theta_n)\rangle
	+ \tfrac{\ubound}{2}\eta^2\| \nabla L(\theta_n)\|^2
	\nonumber\\
	&=L(\theta_n) - \eta(1-\tfrac{\eta \ubound}{2})\|\nabla L(\theta_n)\|^2
	\nonumber
\end{align}
%
We can then parametrize all learning rates for which we can guarantee a positive
decrease by
\begin{align}\label{learning rate restrictions}
	0<\eta=\tfrac{2\alpha}{\ubound}<\tfrac2\ubound \qquad \alpha \in (0,1).
\end{align}
%
This is the same bound we have encountered in the simplified case (\ref{eq:
learning rate restriction (eigenvalue)}) which tells us that there are no more
gains to be made. Plugging the parametrization back into the learning rate we
get
%
\begin{align*}
	L(\theta_{n+1}) - L(\theta_n)
	\le - \tfrac{2}{\ubound}\alpha (1-\alpha)\|\nabla L(\theta_n)\|^2.
\end{align*}
%
The largest decrease is guaranteed by \(\eta=\tfrac{1}{\ubound}\) (\(\alpha=1/2\)).
\begin{remark}
	Recall that this sets the rate of the largest eigenvalue to zero eliminating
	the error in that eigenspace immediately, but it is not the optimal
	discretization for \(\theta\) to move towards a minimum in parameter space.
	But here we do not care about the distance in parameter space but the
	distance between the losses. And favouring the eigenspace with larger
	eigenvalues is results in greater reductions in the loss.
\end{remark}
This results in
%
\begin{align*}
	L(\theta_{n+1}) - L(\theta_n)
	\le - \tfrac{1}{2\ubound}\|\nabla L(\theta_n)\|^2.
\end{align*}
%
\subsubsection{Dead End: Convergence of the Gradient}

Summing over these increments results in a very similar equation to
(\ref{eq: bounded gradient integral})
%
\begin{align*}
	\frac{1}{2\ubound} \sum_{k=0}^{n-1}\|\nabla L(\theta_k)\|^2
	\le L(\theta_0) - L(\theta_n)
	\le L(\theta_0) - \inf_{\theta} L(\theta)
\end{align*}
%
Since \(\tfrac{1}{\ubound}\) is the time increment \(\eta\) between the \(\theta_k\)
we have only lost the factor \(1/2\) in our estimation (\ref{bound increment})
compared to the precise integral version.
In particular we get a convergent average of squared gradients
%
\begin{align*}
	\frac{1}{n} \sum_{k=0}^n \|\nabla L(\theta_k)\|^2
	\le \frac{2 \ubound}{n} (L(\theta_0) - \inf_\theta L(\theta)) \in O(1/n).
\end{align*}
%
It is important to note the "square" part. If the series of unsquared gradient
norms were finite as well, the \(\theta_n\) would be a cauchy sequence
%
\begin{align*}
	\|\theta_n - \theta_m \|
	\le \sum_{k=m}^{n-1} \|\theta_{k+1} - \theta_k\|
	\le \eta \sum_{k=m}^{n-1} \|\nabla L(\theta_k)\|.
\end{align*}
%
This provides us with some intuition how a situation might look like when the
gradient converges but not the sequence of \(\theta_n\). The gradient would
behave something like the harmonic series, as its squares converges and the
\(\theta_n\) would behave like the partial sums of the harmonic series which
behaves like the logarithm in the limit.

It is difficult to formulate an example in finite space, but
%
\begin{align*}
	L(\theta) &= \exp(-\theta) \\
	\dot{\theta} &= -\nabla L(\theta)
\end{align*}
%
implies
%
\begin{align*}
	\theta(t) &= \log(t)\\
	\nabla L(t) &= -\tfrac1t
\end{align*}
%
which provides intuition how "flat" a minima has to be to cause such behavior.
The minimum at \(\infty\) has an infinitely wide basin which flattens out
more and more. If we wanted such an example in a bounded space we would have
to try and coil up such an infinite slope into a spiral, which spirals outwards
to avoid convergence.

Since our example is one dimensional we can also immediately see the "eigenvalue"
%
\begin{align*}
	\nabla^2 L(\theta(t)) = \exp(-\theta(t)) = 1/t
\end{align*}
%
which decreases towards zero, stalling the movement towards the minimum.

\section{Convexity/Convergence of the Loss}

While we do need lower bounds on the second derivative to achieve convergence
of \(\theta(t)\) as motivated in the previous section we can get convergence
of the loss

\begin{theorem}[\citeauthor{nesterovLecturesConvexOptimization2018}]
	Let \(L\) be convex, differentiable with Lipschitz gradient, then Gradient
	Decent with learning rate \(0 < \eta < 2/\ubound\) results in
	\begin{align*}
		L(\theta_n) - \inf_\theta L(\theta)
		\le \frac{2\ubound\|\theta_0 - \theta^*\|}{4 + n\ubound\eta(2-\ubound\eta)}
		\in O(1/n)
	\end{align*}
	with the optimal rate of convergence being achieved for \(\eta=1/\ubound\).
\end{theorem}
\begin{proof}
	See \textcite[Theorem 2.1.14, Corollary
	2.1.2]{nesterovLecturesConvexOptimization2018}
 \end{proof}

It turns out that we can even get convergence of the loss if we only assume
Lipschitz continuity of \(L\) itself. But this form of "Subgradient Decent"
requires a bounded, convex parameter set we project back into if we leave and
a decreasing learning rate. And since we can not guarantee a monotonic decrease
with subgradients, the name "decent" is generally avoided and we have to keep
a running minimum. See \textcite[Section 3.2.3]{nesterovLecturesConvexOptimization2018}
or \textcite[Section 3.1]{bubeckConvexOptimizationAlgorithms2015} for a proper
treatment. The convergence rate is not only dependent on the size of the
parameter set but it also decreases to \(O(1/\sqrt{n})\).

\subsection{Complexity Bound: The Colorization Problem}

\begin{assumption}\label{assmpt: parameter in linear hull of gradients}

\end{assumption}

\begin{theorem}[\citeauthor{nesterovLecturesConvexOptimization2018}]
	For any \(\theta_0\in\reals^d\), any \(k\) such that \(0\le k\le \tfrac12 (d-1)\),
	there exists a convex, smooth function \(f\) with Lipschitz continuous
	derivative such that for any first order method \(\firstOrderMethod\)
	satisfying Assumption~\ref{assmpt: parameter in linear hull of gradients}
	we have
	\begin{subequations}
	\begin{align}
		f(\theta_k) - \inf_\theta f(\theta)
		&\ge \frac{3\ubound \|\theta_0 - \theta^*\|^2}{32(k+1)^2} \\
		\|\theta_k -\theta^*\|^2 
		&\ge \frac18 \|\theta_0 - \theta^*\|^2
	\end{align}
	\end{subequations}
	where \(\theta^* = \arg\min_\theta f(\theta)\).
\end{theorem}
\begin{proof}
	See \textcite[Theorem 2.1.7]{nesterovLecturesConvexOptimization2018}
\end{proof}

\section{Strong Convexity}

\fxnote{If we achieve a contraction towards our minimum, the
function necessarily has to be convex, cf. https://youtu.be/6WeyTUnbwQQ?t=1020}


\section{Heuristics}

\subsection{Adagrad}

\subsection{Adadelta}

\subsection{RMSProp}

%%%%%%%%%%%%%%%%%%%%%%%%%%%%%%%%%%%

\endinput