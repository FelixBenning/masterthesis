% !TEX root = ../Masterthesis.tex

\newcommand{\identity}{\mathbb{I}}
\newcommand{\diag}{\text{diag}}
\newcommand{\rate}{\text{rate}}

\chapter{Gradient Decent}

A step up from zero order methods (such as grid search) are first order methods
using the first derivative (the gradient). This requires a "first order oracle"
which can provide us with \(L(\theta)\) and \(\nabla L(\theta)\) at any point
\(\theta\).

One immediately obvious way to utilize this information is to
incrementally move in the direction of steepest decent, i.e.
%
\begin{align*}
	\theta_{n+1} = \theta_n - \eta\nabla L(\theta_n).
\end{align*}
%
A useful way to look at this equation is to notice that it is the
discretization of an ordinary differential equation (ODE)
%
\begin{align*}
	\dot{\theta}_n \approx \frac{\theta_{n+1} - \theta_n}{\eta}
	= - \nabla L(\theta_n).
\end{align*}
%
Here \(\eta\) is the time delta between discretizations \(t_n\) and \(t_{n+1}\).
This of course implies \(t_n = n\eta\). And for \(\eta\to\infty\), we arrive
at the ODE
%
\begin{align*}
	\dot{\theta}(t) = -\nabla L(\theta(t)).
\end{align*}
%
If you are familiar with Lyapunov functions you will not be surprised by the next
few arguments:
%
\begin{align*}
	L(\theta(t_1)) - L(\theta(t_0))
	&= \int_{t_0}^{t_1} \nabla L(\theta(s)) \cdot \dot{\theta}(s) ds
	= \int_{t_0}^{t_1} -\|\nabla L(\theta(s))\|^2 ds
	\le 0\\
\end{align*}
%
immediately implies
\begin{align*}
	L(\theta(t_0)) \ge L(\theta(t_1)) \ge \dots \ge 0
\end{align*}
which implies convergence. But not necessarily a convergent \(\theta(t)\).

\section{Convergence of the Gradient}

The first thing we might want to ensure, is that the gradient \(\nabla L(\theta(t))\)
actually converges to zero. If we know that the gradient being larger \(\epsilon\)
implies it being larger \(\epsilon/2\) in a \(\delta\) environment around any
such occurrence, we can immediately see that the number of times the gradient
sticks its head above \(\epsilon\) must be finite, as
%
\begin{align}\label{bounded gradient integral}
	0 \le \liminf_{t\to\infty} L(\theta(t))
	=  L(\theta(t_0)) - \int_{t_0}^\infty \|\nabla L(\theta(s))\|^2 ds.
\end{align}
%
So uniform continuity is sufficient for convergence of \(\nabla L(\theta(t))\).
But that still does not ensure that \(\theta(t)\) converges, only that its
derivative \(\dot{\theta}(t) = -\nabla L(\theta(t))\) converges. The logarithm
is an obvious example where this goes wrong.

Additionally we are currently talking about a theoretical ODE where the timestep
is infinitesimal. So let us first return to a discretized gradient. Under the
assumption that \(\nabla L\) is Lipschitz continuous
%
\begin{align*}
	\| \nabla L(\theta) - \nabla L(\tilde{\theta})\| \le K_1 \|\theta-\tilde{\theta}\|
\end{align*}
%
we have for \(\theta_{n+1}=\theta_n-\eta\nabla L(\theta_n)\)
%
\begin{align}
	L(\theta_{n+1})
	&\lxle{\ref{Lipschitz Gradient implies taylor inequality}} L(\theta_n) 
	+ \langle\nabla L(\theta_n),\theta_{n+1} - \theta_n\rangle
	+ \tfrac{K_1}{2} \| \theta_{n+1} - \theta_n\|^2 
	\label{bound increment}\\
	&= L(\theta_n)
	+ \langle\nabla L(\theta_n), -\eta\nabla L(\theta_n)\rangle
	+ \tfrac{K_1}{2}\eta^2\| \nabla L(\theta_n)\|^2
	\nonumber\\
	&=L(\theta_n) - \eta(1-\tfrac{\eta K_1}{2})\|\nabla L(\theta_n)\|^2
	\nonumber
\end{align}
%
this means that for
\begin{align}\label{learning rate restrictions}
	\eta=\tfrac{2\alpha}{K_1},
\end{align}
%
\(\alpha \in (0,1)\) parameterizes all the \(\eta\) for which we can guarantee
a positive decrease
%
\begin{align*}
	L(\theta_{n+1}) - L(\theta_n)
	\le - \tfrac{2}{K_1}\alpha (1-\alpha)\|\nabla L(\theta_n)\|^2
\end{align*}
%
with the optimal choice of \(\eta=\tfrac{1}{K_1}\) (\(\alpha=1/2\)) resulting
in 
%
\begin{align*}
	L(\theta_{n+1}) - L(\theta_n)
	\le - \tfrac{1}{2K_1}\|\nabla L(\theta_n)\|^2.
\end{align*}
%
Summing over these increments results in a very similar equation to
(\ref{bounded gradient integral})
%
\begin{align*}
	\inf_{\theta} L(\theta) \le L(\theta_n) 
	\le L(\theta_0) - \tfrac{1}{2K_1} \sum_{k=0}^{n-1}\|\nabla L(\theta_k)\|^2
\end{align*}
%
Since \(\eta=\tfrac{1}{K_1}\) is the time increment between the \(\theta_k\)
we have only lost the factor \(1/2\) in our estimation (\ref{bound increment})
compared to the precise integral version.

In particular we get
%
\begin{align*}
	\tfrac{n}{2K_1}
	\underbrace{\tfrac{1}{n} \sum_{k=0}^n \|\nabla L(\theta_k)\|^2}_{
		=: \overline{\|\nabla L(\theta_n)\|^2}
	}
	\le \tfrac{1}{2K_1}\sum_{k=0}^{n-1} \|\nabla L(\theta_k)\|^2 
	\le L(\theta_0) - \inf_\theta L(\theta)
\end{align*}
%
which implies that the average gradient square converges with \(n\) to zero
%
\begin{align*}
	\overline{\|\nabla L(\theta_n)\|^2}
	\le \tfrac{1}{n}2 K_1 (L(\theta_0) - \inf_\theta L(\theta)) \in O(\tfrac{1}{n}).
\end{align*}
%
It is important to note the "square" part of the gradient squares. If we had the
series of unsquared gradient norms being finite as well, we would have
\(\theta_n\) being a cauchy sequence
%
\begin{align*}
	\|\theta_n - \theta_m \|
	\le \sum_{k=m}^{n-1} \|\theta_{k+1} - \theta_k\|
	\le \tfrac{1}{2K_1} \sum_{k=m}^{n-1} \|\nabla L(\theta_k)\|.
\end{align*}
%
This provides us with some intuition how a situation might look like when the
gradient converges but not the sequence of \(\theta_n\). The gradient would
behave something like the harmonic series, as its squares converges and the
\(\theta_n\) would behave like the partial sums of the harmonic series which
behaves in the limit like the logarithm.

It is difficult to formulate an example in finite space, but
%
\begin{align*}
	L(\theta) &= \exp(-\theta) \\
	\dot{\theta} &= -\nabla L(\theta)
\end{align*}
%
implies
%
\begin{align*}
	\theta(t) &= \log(t)\\
	\nabla L(t) &= -\tfrac1t
\end{align*}
%
which provides intuition how "flat" a minima has to be to cause such behavior.
The minimum at \(\infty\) has an infinitely wide basin which flattens out
more and more. If we wanted such an example in a bounded space we would have
to try and coil up such an infinite slope into a spiral, which spirals outwards
to avoid convergence.

\section{All Roads Lead to Convexity}

Before we further tighten our restrictions on \(L\) to ensure convergence, let
me point out the following lemma
%
\begin{lemma}\label{lem: lipschitz and bounded derivative}
	If \(f\) is differentiable, then derivative \(\nabla f\) is
	bounded (w.r.t. the operator norm) by constant \(K\) iff the function
	\(f\) is Lipschitz continuous with Lipschitz constant \(K\).
\end{lemma}
\begin{proof}
	Lipschitz continuity is implied by the mean value theorem
	\begin{align*}
		\|f(x_1) - f(x_0)\|
		&\le \|\nabla f(x_0 + \xi(x_1-x_0))\| \|x_1- x_0\|\\
		&\le K \|x_1-x_0\|.
	\end{align*}
	%
	The opposite direction is implied by
	%
	\begin{align*}
		\|\nabla f(x_0)\|
		&\equiv \sup_{v} \frac{\|\nabla f(x_0)v\|}{\|v\|}
		= \sup_{v} \lim_{v\to 0}\frac{\|\nabla f(x_0)v\|}{\|v\|}\\
		&\lxle{\Delta}\sup_{v} \lim_{v\to 0}
		(
			\underbrace{
				\tfrac{\|f(x_0 + v) - f(x_0) \|}{\|v\|}
			}_{
				\le K
			}
			+ \underbrace{
				\tfrac{\|\nabla f(x_0)v + f(x_0) - f(x_0 +v)\|}{\|v\|}
			}_{
				\to 0 \text{ (derivative definition)}
			}
		)
	\end{align*}
	%
	where we have used the scalability of the norm to multiply \(v\) with a
	decreasing factor both in the numerator and denominator in order to introduce
	the limit.
\end{proof}

Lipschitz continuity of \(\nabla L\) is thus equivalent to boundedness of the
Hesse matrix \(\nabla^2 L\). And since the Hesse matrix is symmetric and thus
diagonizable with an orthonormal basis, the operator norm of it is equal to its
largest absolute eigenvalue. The Lipschitz constant thus bounds the speed at
which the gradient changes in the direction of an eigenvector corresponding to
this largest absolute eigenvalue.

The slower the gradient changes, the further we can walk before we have to
take another measurement. For this reason the optimal learning rate is
inversely proportional to the Lipschitz constant.

The Lipschitz constant \(K_1\) restricts how sharp the minima can be, 
this ensures that we do not overshoot it and bounce around aimlessly with a 
learning rate that is much too high. But as we have seen above, a very flat
minima might decelerate gradient decent so much that our parameter \(\theta_n\)
never converges.

So it is a natural conclusion to bound the Hesse matrix from below as well.
Now if there is a local minima somewhere, then it can not have negative
eigenvalues in its Hessian. Those would imply a direction in which the gradient
\emph{decreases} below zero and thus that the minima is in fact not a minima.
So if we assume that the absolute eigenvalues are bounded from below, assume
that \emph{a} local minima exists, and assume that the second derivative is
continuous, then the positive eigenvalues around that local minima can
never become negative due to the lower bound on the absolute eigenvalues and
continuity. In other words: We are assuming strong, global convexity. Which of
course is anything but a natural assumption.

Alternatively one could make this assumption only about the local minimum and
make continuity assumptions about the second derivative, then there is a local
space around the minimum where this bound holds and we can get convergence
statements conditioned on starting in this local space.

Since neither option resembles real applications, we will focus on the intuition
similarly to \textcite{gohWhyMomentumReally2017} instead of mathematical rigor.
For a proper treatment of the local case see \textcite[Theorem
1.2.4]{nesterovLecturesConvexOptimization2018} and for the global case see
\textcite[Theorem 2.1.15]{nesterovLecturesConvexOptimization2018}.

\fxnote{Different angle: If we achieve a contraction towards our minimum, the
function necessarily has to be convex, cf. https://youtu.be/6WeyTUnbwQQ?t=1020}


\section{Visualizing the 2nd Taylor Approximation}
If we assume our Hesse matrix \(\nabla^2 L(\theta)\) to be positive definite
(all eigenvalues positive), then for a second order Taylor approximation
%
\begin{align*}
	L(\theta+v) \approx T_2L(\theta+v)
	= L(\theta) + v^T \nabla L(\theta) + \tfrac12 v^T \nabla^2 L(\theta) v
\end{align*}
%
we can find the minimum 
\begin{align*}
	\hat{v} = -(\nabla^2 L(\theta))^{-1}\nabla L(\theta)
\end{align*}
by setting the first derivative to zero
%
\begin{align*}
	\nabla T_2L(\theta+v) = \nabla L(\theta) + \nabla^2L(\theta) v \xeq{!} 0.
\end{align*}
%
This minimum allows us not only to rewrite the taylor derivative
%
\begin{align*}
	\nabla T_2L(\theta+v) = \nabla^2 L(\theta)(v-\hat{v}),
\end{align*}
%
but also the original taylor approximation
%
\begin{align*}
	T_2L(\theta+v)
	&= L(\theta) - v^T \nabla^2 L(\theta) \hat{v} + \tfrac12 v^T \nabla^2 L(\theta) v \\
	&= \underbrace{L(\theta) - \tfrac12 \hat{v} \nabla^2 L(\theta) \hat{v}}_{=: \alpha \text{ (const)}}
	+ \tfrac12 (v-\hat{v})^T \nabla^2 L(\theta)(v-\hat{v}).
\end{align*}
%
To get absolute instead of relative coordinates to \(\theta\) we set
%
\begin{align*}
	\hat{\theta} := \theta + \hat{v} = \theta -(\nabla^2 L(\theta))^{-1}\nabla L(\theta),
\end{align*}
%
and obtain the notion of a paraboloid centered around \(\hat{\theta}\)
%
\begin{align}\label{paraboloid approximation of L}
	L(y) = \tfrac12 (y- \hat{\theta}) \nabla^2 L(\theta) (y-\hat{\theta}) + \alpha + o(\|y-\theta\|^2)
\end{align}
%
To fully appreciate Figure~\ref{fig: 2d paraboloid}, we only need to realize
that the diagonizability of \(\nabla^2 L(\theta)\)
%
\begin{align}\label{diagnalization of the Hesse matrix}
	V \nabla^2 L(\theta) V^T
	= \diag(\lambda_1,\dots,\lambda_d), \qquad V=(v_1,\dots, v_d)
\end{align}
%
\begin{wrapfigure}{O}{0.65\textwidth}
	\centering
	\def\svgwidth{0.65\textwidth}
	\input{media/contour.pdf_tex}
	\caption{Assuming \(\hat{\theta}=0\), \(\lambda_1=1, \lambda_2=2\), \(v_1=(\sin(1), \cos(1))\)}
	\label{fig: 2d paraboloid}
\end{wrapfigure}
%
implies that once we have selected the center \(\hat{\theta}\) and the direction of
one eigenspace in two dimensions, the other eigenspace has to be
orthogonal to the first one which fully determines its direction at this point. 

Before we move on I want to briefly mention that we only really needed the
positive definiteness of \(\nabla^2 L(\theta)\) to make it invertible, so that
\(\hat{v}\) is well defined. If it was not positive definite but still invertible,
then \(\hat{v}\) would not be a minimum but all the other arguments would still
hold.
In that case the eigenvalues might be negative as well as positive, which would
represent a saddle point, or all negative which would represent a maximum.

\section{From Eigenvalues to Convergence Rate}

Using the representation (\ref{paraboloid approximation of L}) of \(L\) we
can write the gradient at \(\theta\) as
%
\begin{align*}
	\nabla L(\theta)
	=  \nabla^2 L(\theta)(\theta-\hat{\theta})
	\ (= -\nabla^2 L(\theta)\hat{v})
\end{align*}
%
But note that \(\hat{\theta}\) depends on \(\theta\), so we need index both
by \(n\) to rewrite gradient decent
%
\begin{align*}
	\theta_{n+1} &= \theta_n - \eta\nabla L(\theta_n)\\
	&= \theta_n - \eta\nabla^2 L(\theta_n)(\theta_n - \hat{\theta}_n).
\end{align*}
%
Subtracting \(\hat{\theta}_n\) from both sides we obtain the following
transformation 
%
\begin{align*}
	\theta_{n+1} - \hat{\theta}_n
	&= (\identity - \eta\nabla^2 L(\theta_n) ) (\theta_n - \hat{\theta}_n).
\end{align*}
%
Taking a closer look at this transformation matrix we can see
%
\begin{align*}
	\identity - \eta\nabla^2 L(\theta_n)
	&= V(\identity - \eta\cdot\diag(\lambda_1,\dots,\lambda_d) )V^T \\
	&= V\cdot\diag(1-\eta\lambda_1, \dots,1-\eta\lambda_d)V^T.
\end{align*}
%
Now if we assume like \textcite{gohWhyMomentumReally2017}, that the second
taylor approximation is accurate and thus that \(\nabla^2 L(\theta)=H\) is a
constant, then \(\hat{\theta}_n = \theta^*\) is the real minimum and we get
%
\begin{align}
	\theta_n - \theta^*
	= V\cdot\diag[(1-\eta\lambda_1)^n,\dots,(1-\eta\lambda_d)^n] V^T (\theta_0 - \theta^*)
\end{align}
%
by induction. Decomposing the difference into the eigenspaces of \(H\) we can 
see that each component scales exponentially on its own 
%
\begin{align*}
	\langle \theta_n -\theta^*, v_i\rangle = (1-\eta\lambda_i)^n \langle \theta_0 - \theta^*, v_i\rangle
\end{align*}
%
This is beautifully illustrated with interactive graphs in
\citetitle{gohWhyMomentumReally2017} by \citeauthor{gohWhyMomentumReally2017}.

\subsection{A Note on negative eigenvalues}

Now if we have negative eigenvalues, then \(1-h\lambda_i\) would be greater
than one which repels this component away from \(\theta^*\). This is a good
thing, since \(\theta^*\) is a maximum in this component. This means we will
walk down local minima and saddle points no matter the learning rate \emph{if}
this component was not zero to begin with. In case of a maximum this would mean
that we would start right on top of the maximum, in case of a saddle point
it implies starting on the rim such that one can not roll down either side.

But since we are ultimately interested in stochastic loss functions, the
stochasticity would push us off these narrow equilibria so being right on
top of them is of little concern. But being close to zero in such a component
still means slow movement away, since we are multiplying by it. This is a
common explanation for the observation of temporary plateaus in machine learning
which "suddenly fall off" once the exponential factor ramps up and causes a
sharp drop in the loss (Figure \ref{fig: visualize saddlepoint gd}).

\begin{figure}[h]
	\centering
	\def\svgwidth{1\textwidth}
	\input{media/visualize_gd.pdf_tex}
	\caption{Start=\(0.001v_1+4v_2\), \(\lambda_1=-1, \lambda_2=2\), learning rate\(=0.8\)}
	\label{fig: visualize saddlepoint gd}
\end{figure}

\subsection{Assuming Convexity}

Let us consider strictly positive eigenvalues now and assume that our
eigenvalues are already sorted
%
\begin{align}
	0 < \lambda_1 \le \dots \le \lambda_d.
\end{align}
%
Then for positive learning rates all exponentiation bases are smaller than one
%
\begin{align*}
	1-\eta\lambda_d \le \dots \le 1-\eta\lambda_1 \le 1.
\end{align*}
%
But to ensure convergence we also need that all of them are larger than \(-1\).
This leads us to the condition
\begin{align}\label{eq: learning rate restriction (eigenvalue)}
	0< \eta < 2/\lambda_d
\end{align}
As you might recall
 this is the exact same condition we have already seen (cf.
(\ref{learning rate
restrictions})) if you also recall Lemma \ref{lem: lipschitz and bounded
derivative}.
Since that lemma states that the Lipschitz constant \(K_1\) of the gradient was the
same as the operator norm of \(\nabla^2 L\) which happens to be the largest 
absolute eigenvalue \(\lambda_d\).

Now in that situation we tried to make the gradient converge as quickly as
possible and found that \(\eta=1/\lambda_d\) would provide the best bounds.

This selection of the learning rate greedily reduces the exponentiation factor
of the eigenspace corresponding to the largest eigenvalue to zero. Ensuring
that the component which contributes the most to the size of the gradient is
reduced to nothing immediately.

But if we want to maximize the convergence rate of \(\theta_n\), this is not
the best selection.

And we can reduce the exponentiation bases of the other eigenspaces if we
increase the learning rate further, getting them closer to zero. At this point
we are of course overshooting zero with the largest learning rate, so we only
continue with this until we get closer to \(-1\) with the largest eigenvalue
than we are to \(1\) with the smallest. The largest absolute exponentiation base
is the convergence rate
%
\begin{align*}
	\rate(\eta)=\max_{i} |1-\eta\lambda_i| = \max\{|1-\eta\lambda_1|, |1-\eta\lambda_d|\}.
\end{align*}
%
Which is minimized when
%
\begin{align*}
	1-\eta\lambda_1 = \eta\lambda_d -1,
\end{align*}
%
implying
%
\begin{align*}
	\eta^* = \frac{2}{\lambda_1 + \lambda_d}.
\end{align*}
%
If \(\lambda_1\) is much smaller than \(\lambda_d\), this leaves \(\eta\)
at the upper end of the interval in (\ref{eq: learning rate restriction
(eigenvalue)}). And the convergence rate
%
\begin{align*}
	\rate(\eta^*)
	&= \frac{1-\eta^* \lambda_1}{2} + \frac{\eta^*\lambda_d -1}{2}
	= \frac{\lambda_d - \lambda_1}{\lambda_d + \lambda_1}\\
	&= \frac{\lambda_d/\lambda_1-1}{\lambda_d/\lambda_1+1}
\end{align*}
\fxnote{use different representation (only one kappa)}
%
becomes close to one if the condition number \(\kappa:=\lambda_d/\lambda_1\)
is large.
If all the eigenvalues are the same, the condition number becomes one and the
rate is zero, implying instant convergence. This is not surprising if we recall
our visualization in Figure \ref{fig: 2d paraboloid}. When the eigenvalues are
the same, the contours are concentric circles and the gradient points right at
the center.

\fxnote{strong convexity and lipshitz continuity of gradient as lower and
upper bound, sketch convergence proof using that instead of eigenvalues}

\subsection{Momentum}

\subsection{Optimal Methods/Bounds/Coloring/Value function learning}

\subsection{Nestrov Optimal Method/Nestrov Momentum}

\subsection{Newton}

\subsection{Conjugate Gradient}

\subsection{Runge Kutta}


%%%%%%%%%%%%%%%%%%%%%%%%%%%%%%%%%%%

\endinput