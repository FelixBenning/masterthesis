% !TEX root = ../Masterthesis.tex

\chapter{Momentum}

To understand why the convergence rate is poor when the condition
number is high, we can visualize a high ratio of the lowest to the highest
eigenvalue as a narrow ravine. The gradient points in the direction of the
strongest decent which is mostly to the opposite side of the ravine and only slightly
along its length. This causes our iterate to bounce back and forth between
the walls of the ravine.
%
\begin{figure}[h]
	\centering
	\def\svgwidth{1\textwidth}
	\input{media/visualize_bad_contitioning.pdf_tex}
	\caption{Momentum reduces fluctuations and converges faster}
	\label{fig: visualize bad conditioning}
\end{figure}

As a fix it seems appropriate to average the gradients in some sense, to
cancel out the opposing jumps and go straight down the ravine. In other words
we want to build momentum. Now if we move according to the sum (integral) of
the gradients, then our velocity stops being equal to the gradient but instead
becomes the antiderivative of the gradient.

So instead of setting our gradient equal to the velocity like in (\ref{eq:
velocity is gradient}), we want to set the acceleration equal to our gradient
%
\begin{align*}
	\ddot{\weights} = -\nabla \Loss(\weights).
\end{align*}
%
But without friction we are going to massively overshoot the minimum, so we are
also going to add a "friction force" inversely proportional to our current
velocity
%
\begin{align}\label{eq: acceleration is gradient + friction}
	\ddot{\weights} = -\nabla \Loss(\weights) - \friction \dot{\weights}.
\end{align}
%
The standard way to discretize a second order ODE is to convert it into a first
order ODE
%
\begin{align*}
	\dot{y} := \begin{pmatrix}
		\dot{\weights}\\
		\ddot{\weights}
	\end{pmatrix}
	= \begin{pmatrix}
		\dot{\weights} \\
		-\nabla \Loss(\weights) - \friction \dot{\weights}
	\end{pmatrix}
	=: g\Big(\begin{pmatrix}
		\weights \\
		\dot{\weights}
	\end{pmatrix}\Big)
	= g(y).
\end{align*}
%
Which allows us to naively discretize our ODE with the Euler discretization
\fxnote{whitespace issue with subequations}
%
\begin{subequations}
\begin{align}
	\weights_{n+1} &= \weights_n + \lr \momentum_n \label{eq: naive momentum move}\\
	\momentum_{n+1} &= \momentum_n + \lr [-\nabla \Loss(\weights_n) - \friction \momentum_n]
	\label{eq: naive momentum}\\ \nonumber
	&= (1-\lr\friction)\momentum_n - \lr\nabla \Loss(\weights_n).
\end{align}
\end{subequations}
%
Here we use \(\momentum\) to denote the momentum (velocity \(\dot{\weights}\)
assuming unit mass).
If we plug the second equation (\ref{eq: naive momentum}) into the first
equation (\ref{eq: naive momentum move}) we get
%
\begin{align*}
	\weights_{n+1}
	&= \weights_n + \lr [(1-\lr\friction)\momentum_{n-1} - \lr\nabla \Loss(\weights_{n-1})].
\end{align*}
%
This means we are using gradient information from \(\weights_{n-1}\) to update
\(\weights_{n+1}\). If we instead use the most up to date information
\(\momentum_{n+1}\) instead of \(\momentum_n\) for the \(\weights_{n+1}\) update,
we get the well known \emph{heavy ball method} (momentum method) first proposed
by \textcite{polyakMethodsSpeedingConvergence1964} and wonderfully illustrated
by \textcite{gohWhyMomentumReally2017}.

\begin{definition}[Momentum Method]
	\begin{subequations}
	\begin{align}
		\weights_{n+1} &= \weights_n + \lr \momentum_{n+1} \label{eq: momentum move}\\
		\momentum_{n+1} &= (1-\lr\friction)\momentum_n - \lr\nabla \Loss(\weights_n)
		\label{eq: momentum}
	\end{align}
	\end{subequations}
	%
	An equivalent formulation obtained by plugging (\ref{eq: momentum}) into
	(\ref{eq: momentum move}) but using (\ref{eq: momentum move}) for
	\(\momentum_n\) is
	%
	\begin{align}\label{eq: flat momentum}
		\weights_{n+1}
		&= \weights_n
		+ \underbrace{(1-\lr\friction)}_{
			=:\momCoeff
		}(\weights_n - \weights_{n-1})
		- \underbrace{\lr^2}_{=:\lrSq}\nabla \Loss(\weights_n)
	\end{align}
	In particular we can set "the momentum coefficient" \(\momCoeff\) to zero to
	obtain gradient decent again. This is of course an artifact of our
	discretization since \(\lr\to0\) would never allow \(\momCoeff\) to be zero
	in the limit. But actual implementations often use this
	(\(\momCoeff,\lrSq\))-parametrization and thus treat gradient decent as a
	special case.
\end{definition}
%
Nesterov's momentum is even more aggressive: Instead of using \(p_{n+1}\) for
the \(\weights_{n+1}\) update, it considers the certain "momentum move"
to calculate an intermediate position \(y_{n+1}\)
%
\begin{align*}
	\weights_{n+1}
	&= \weights_n + \overbrace{\lr [(1-\lr\friction)\momentum_n}^{\text{"momentum move"}}
	- \lr\nabla \Loss(\weights_n)] \\
	&= y_{n+1} - \lrSq \nabla \Loss(\weights_n).
\end{align*}
%
It then uses that intermediate position to calculate the gradient instead of the
previous position \(\weights_n\).
%
\fxnote{Illustrate Nesterov's Momentum}
\begin{definition}[Nesterov's Momentum]
	\begin{subequations}
	\begin{align}
		\weights_{n+1} &= \weights_n + \lr \momentum_{n+1} \label{eq: nesterov momentum move}\\
		\momentum_{n+1}
		&= (1-\lr\friction)\momentum_n
		- \lr\nabla L[\weights_n + \lr(1-\lr\friction)\momentum_n]
		\label{eq: nesterov momentum}
		\\ \nonumber
		&= \momCoeff\momentum_n
		- \lr\nabla L[\underbrace{
			\weights_n + \momCoeff(\weights_n - \weights_{n-1})
		}_{= y_{n+1}}]
	\end{align}
	\end{subequations}
	%
	Discarding the momentum term and solely using the intermediate position
	results in the simplified version
	%
	\begin{subequations} \label{eq: nesterov intermediate position version}
	\begin{align}
		\weights_{n+1} &= y_{n+1} - \lrSq \nabla \Loss(y_{n+1})\\
		y_{n+1}&= \weights_n + \momCoeff(\weights_n - \weights_{n-1})
	\end{align}
	\end{subequations}
	dropping the intermediate position as well results in the analog to (\ref{eq:
	flat momentum})
	\begin{align}
		\weights_{n+1} &= \weights_n + \momCoeff(\weights_n - \weights_{n-1})
		- \lrSq \nabla \Loss(\weights_n + \momCoeff(\weights_n - \weights_{n-1}))
	\end{align}
\end{definition}
%
\begin{remark}\fxwarning{too informal?}
	This method also known as "Nesterov's Accelerated Momentmum" is said to date
	back to \textcite{nesterovMethodSolvingConvex1983}.  But not only is the
	original text in Russian, it also difficult to find on the internet.
	Fortunately, \citeauthor{nesterovMethodSolvingConvex1983} wrote textbooks,
	\citetitle{nesterovLecturesConvexOptimization2018}
	(\citeyear{nesterovLecturesConvexOptimization2018}) being the most recent.
	Unfortunately, the chapter 2.2 on optimal methods provides barely any
	intuition at all and it is advisable to consult other sources (e.g.
	\textcite{dontlooWhatDifferenceMomentum2016}). If you
	want to recognize this scheme in \textcite{nesterovLecturesConvexOptimization2018}
	compare (\ref{eq: nesterov intermediate position version}) with Nesterov's
	Constant Step Scheme III (2.2.22).
 \end{remark}

\section{Convergence Rate}

Following the arguments from Section~\ref{sec: visualize gd} in particular
(\ref{eq: hesse representation of gradient}) we can rewrite the momentum
method as
\begin{align*}
	\begin{pmatrix}
		\weights_n - \hat{\weights}_n \\
		\weights_{n+1} - \hat{\weights}_n
	\end{pmatrix}
	&=
	\begin{pmatrix}
		\weights_n - \hat{\weights}_n \\
		\weights_n - \hat{\weights}_n + \momCoeff (\weights_n - \weights_{n-1})
		- \lrSq \nabla\Loss^2(\weights_n)(\weights_n -\hat{\weights}_n)
	\end{pmatrix}\\
	&=
	\begin{pmatrix}
		0\identity_d & \identity_d \\
		-\momCoeff\identity_d & (1+\momCoeff)\identity_d -\lrSq \nabla^2\Loss(\weights_n)
	\end{pmatrix}
	\begin{pmatrix}
		\weights_{n-1} - \hat{\weights}_n \\
		\weights_n - \hat{\weights}_n
	\end{pmatrix}
\end{align*}
%
For readability I will now omit the identity matrix \(\identity\) from the
block matrices which are just a constant multiplied by an identity matrix.
Using the digitalization of the hesse matrix (\ref{eq: diagnalization of the
Hesse matrix}) again, we get
%
\begin{align*}
	&\begin{pmatrix}
		0 & 1 \\
		-\momCoeff & 1+\momCoeff -\lrSq \nabla^2\Loss(\weights_n)
	\end{pmatrix}\\
	&=
	\begin{pmatrix}
		V & 0 \\
		0 & V
	\end{pmatrix}	
	\begin{pmatrix}
		0 & 1 \\
		-\momCoeff &
		\diag(1+\momCoeff -\lrSq\hesseEV_1, \dots, 1+\momCoeff -\lrSq\hesseEV_d)
	\end{pmatrix}
	\begin{pmatrix}
		V & 0 \\
		0 & V
	\end{pmatrix}^T
\end{align*}
%
Reordering the eigenvalues to
%
\begin{align*}
	\begin{pmatrix}
		v_1 & 0 & \cdots & v_d & 0 \\
		0 & v_1 & \cdots & 0 & v_d
	\end{pmatrix}
\end{align*}
%
reorders the transformation matrix of the eigenspace to
%
\begin{align*}
	\Sigma := \begin{pmatrix}
		0 & 1 \\
		-\momCoeff & 1+\momCoeff - \lrSq\hesseEV_1 & \\
		& & \ddots & \\
		& & & 0 & 1 \\
		& & & -\momCoeff & 1+\momCoeff - \lrSq\hesseEV_d \\
	\end{pmatrix}
	= \begin{pmatrix}
		\Sigma_1 \\
		& \ddots\\
		&& \Sigma_d
	\end{pmatrix}
\end{align*}
%
So in contrast to Section~\ref{sec: visualize gd} we do not get a diagonal
matrix immediately. To achieve a similar eigenvalue analysis as in
Section~\ref{sec: visualize gd} we first have to determine the eigenvalues
\(\sigma_{i1},\sigma_{i2}\) of each \(\Sigma_i\) and then ensure that
\begin{align}
	\max_{i=1,\dots, d} \max\{|\sigma_{i1}|,|\sigma_{i2}|\} < 1.
\end{align}
Using the p-q formula on the characteristic polynomial of \(\Sigma_i\)
results in 
\begin{align*}
	\sigma_{i1/2}
	= \tfrac12 \left(
		1+\momCoeff-\lrSq\hesseEV_i
		\pm \sqrt{(1+\momCoeff-\lrSq\hesseEV_i)^2 - 4\momCoeff}
	\right).
\end{align*}
%
The analysis of these eigenvalues is very technical and can be found in the
appendix. We will only cover the result here.

\begin{figure}[h]
	\centering
	\def\svgwidth{1\textwidth}
	\input{media/annotated_heavy_ball_rates.pdf_tex}
	\caption{
		A heat plot of the absolute value of \(\max\{|\sigma_1|,|\sigma_2|\}\).
		Strictly speaking the "Monotonic" and "Oscillation" area should not
		extend into the area of negative \(\momCoeff\) since \(\sigma_{1/2}\)
		have opposite signs there, but it still describes the behavior of the
		dominating eigenvalue.
	}
	\label{fig: annotated heavy ball rates}
\end{figure}

\begin{theorem}[\cite{qianMomentumTermGradient1999}]
	\label{thm: momentum - stable set of parameters}
	Let
	\begin{align*}
		\sigma_{1/2}
		= \tfrac12 \left(
			1+\momCoeff-\lrSq\hesseEV \pm \sqrt{(1+\momCoeff-\lrSq\hesseEV)^2 - 4\momCoeff}
		\right)
	\end{align*}
	then 
	\begin{enumerate}
		\item \(\max\{|\sigma_1|,|\sigma_2|\}<1\) if and only if
		\begin{align*}
			0<\lrSq\hesseEV < 2(1+\momCoeff) \qquad \text{and} \qquad |\momCoeff|<1
		\end{align*}
		\item The complex case can be characterized by either
		\begin{align*}
			0<(1-\sqrt{\lrSq\hesseEV})^2 < \momCoeff < 1
		\end{align*}		
		or alternatively \(\momCoeff>0\) and
		\begin{align*}
			(1-\sqrt{\momCoeff})^2 < \lrSq\hesseEV < (1+\sqrt{\momCoeff})^2,
		\end{align*}
		for which we have \(|\sigma_1|=|\sigma_2|=\sqrt{\momCoeff}\).
		
		As complex
		eigenvalues imply a rotation, this can be viewed as rotating the distance
		to the minimum of the quadratic function into the momentum and back like 
		a pendulum where the friction causes it to eventually end up in the
		minimum. Looking at the distance to the minimum only, we will therefore
		observe a sinus wave ("Ripples").
		\item In the real case we have \(\sigma_1>\sigma_2\) and
		\begin{align}\label{eq: when does sigma_1 or sigma_2 dominate?}
			\max\{|\sigma_1|, |\sigma_2|\} = \begin{cases}
				|\sigma_1|=\sigma_1 & \lrSq\hesseEV < 1+\momCoeff \\
				|\sigma_2|=-\sigma_2 & \lrSq\hesseEV \ge 1+\momCoeff.
			\end{cases}
		\end{align}
		Restricted to \(1>\momCoeff>0\) this results in two different	
		behaviors. For
		\begin{align*}
			0<\lrSq\hesseEV \le (1-\sqrt{\momCoeff})^2 < 1+\momCoeff
		\end{align*}
		we have \(1>\sigma_1 > \sigma_2 > 0\) which results in a monotonic
		linear convergence ("Monotonic" case). For
		\begin{align*}
			1+\momCoeff < (1+\sqrt{\momCoeff})^2\le \lrSq\hesseEV < 2(1+\momCoeff)
		\end{align*}
		on the other hand, we get \(-1 < \sigma_2 < \sigma_1 < 0\) which implies 
		an oscillating convergence ("Oscillation"). In contrast to the "Ripples"
		we switch the side of the distance to the minimum in every iteration.
	\end{enumerate}
\end{theorem}
\begin{proof}
	Appendix Theorem~\ref{thm-appdx: momentum - stable set of parameters}
\end{proof}

\section{Complexity Bounds}

\section{Adam \& Nadam}

%%%%%%%%%%%%%%%%%%%%%%%%%%%%%%%%%%%

\endinput
