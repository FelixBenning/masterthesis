% !TEX root = Masterthesis.tex

%% AMS Theorem Environments
\theoremstyle{plain}% default
\newtheorem{prop}{Proposition}[section]
\newtheorem{lemma}[prop]{Lemma}
\newtheorem{corollary}[prop]{Corollary}
\newtheorem{theorem}[prop]{Theorem}

\theoremstyle{definition}
\newtheorem{definition}[prop]{Definition}
\newtheorem{example}[prop]{Example}
\newtheorem{assumption}[prop]{Assumption}

\theoremstyle{remark}
\newtheorem{remark}[prop]{Remark}

%%%%%%%%%%%%%%%%%%%%% ARROWS %%%%%%%%%%%%%%%%%%%%%%%%%%%%%
% relation with explanation 
% Usage:
%  \begin{algin}
%    Statement &\lximplies{Reason} Other Statement \\
%    &\implies ...
%  \end{align}

\makeatletter
%(https://tex.stackexchange.com/questions/467779/alignment-of-implication-arrows-with-text-on-top)
% % % % % % % % % % % % % % % % % % % % % % % % % % % % % % % % % % % % % % %
\newcommand*{\sm@shstack}[3][lr]{
	\mathrel{% makes the resulting stack a relation class (like =,<,>,...)
		\smashoperator[#1]{\mathop{#2}^{#3}}}
}

\newcommand*{\lx@rrow}[2]{ %uses leftsides smashoperator thus ignoring overlap towards the left
	\sm@shstack[l]{#2}{#1}
}% use when aligning to the left
\newcommand*{\x@rrow}[2]{%uses no smashoperator, but adds back stretchability of \implies to the result contrary to \stackrel
	\;\mkern-\thickmuskip\mathop{#2}\limits^{#1}\;\mkern-\thickmuskip
}% use in \(\) and as individual operator without aligning neccessities


\newcommand*{\lximplies}[1]{\lx@rrow{#1}{\implies}} %%%%% =>
\newcommand*{\lximpliedby}[1]{\lx@rrow{#1}{\impliedby}} %%%%%% <=
\newcommand*{\lxiff}[1]{\lx@rrow{#1}{\iff}} %%%%%%%%%%%%% <=>
\newcommand*{\lxeq}[1]{\lx@rrow{#1}{=}} %%%%%%%%%% =
\newcommand*{\lxle}[1]{\lx@rrow{#1}{\le}}
\newcommand*{\lxge}[1]{\lx@rrow{#1}{\ge}}

\newcommand*{\ximplies}[1]{\x@rrow{#1}{\implies}} %%%%% =>
\newcommand*{\ximpliedby}[1]{\x@rrow{#1}{\impliedby}} %%%%%% <=
\newcommand*{\xiff}[1]{\x@rrow{#1}{\iff}} %%%%%%%%%%%%% <=>
\newcommand*{\xeq}[1]{\x@rrow{#1}{=}} %%%%%%% =
\newcommand*{\xle}[1]{\x@rrow{#1}{\le}} %%%%%%%
\newcommand*{\xge}[1]{\x@rrow{#1}{\ge}} %%%%%%%
\makeatother

%%%%%%%%%%%%%%%%%%%%%%%%%%%%%%%%%%%%%%%%%%%%%%%%%%%%%%%

\newcommand{\identity}{\mathbb{I}}
\newcommand{\lbound}{\mu}
\newcommand{\ubound}{L}
\newcommand{\diag}{\text{diag}}
\newcommand{\rate}{\text{rate}}
\newcommand{\reals}{\mathbb{R}}
\newcommand{\naturals}{\mathbb{N}}
\newcommand{\firstOrderMethod}{\mathcal{M}}
\newcommand{\linSpan}{\text{span}}
\newcommand{\dimension}{d}
\newcommand{\Loss}{\mathcal{L}}
\newcommand{\loss}{l}
\newcommand{\weights}{\theta}
\newcommand*{\model}[1][\weights]{f_{#1}}
\newcommand{\minimum}{\weights_*}
\newcommand{\lr}{\text{h}}
\newcommand{\condition}{\kappa}
\newcommand{\stdBasis}{e}
\newcommand{\graph}{G}
\newcommand{\graphLaplacian}{A_\graph}
\newcommand{\edges}{E}
\newcommand{\vertices}{V}
\newcommand*{\lipGradientSet}[2][1,1]{\mathcal{F}^{#1}_{#2}}
\newcommand*{\strongConvex}[3][1,1]{\mathcal{S}^{#1}_{{#2},{#3}}}
\newcommand*{\sequenceSpace}[1][2]{\ell^{#1}}
\newcommand{\momentum}{p}
\newcommand{\friction}{\mu}
\newcommand{\dist}{\mathcal{D}}
\newcommand{\E}{\mathbb{E}}
\renewcommand{\Pr}{\mathbb{P}}
\newcommand{\lipConst}{K}
\newcommand{\lossError}{\epsilon}
\newcommand{\bergmanDiv}[1]{D^{(B)}_{#1}}
\newcommand{\martIncr}{\delta M}
\newcommand{\martingale}{M}
\newcommand{\filtration}{\mathcal{F}}
\newcommand{\indep}{\mathrel{\;\perp \!\!\! \perp\;}}
